% \iffalse meta-comment
%
% Copyright (C) 2018-2019 by Iydon Liang <liangiydon AT gmail.com>
%
% This file may be distributed and/or modified under the
% conditions of the LaTeX Project Public License, either version 1.3c
% of this license or (at your option) any later version.
% The latest version of this license is in
%    https://www.latex-project.org/lppl.txt
% and version 1.3c or later is part of all distributions of LaTeX
% version 2005/12/01 or later.
%
%<*internal>
\iffalse
\fi
\begingroup
  \def\nameoflatex{LaTeX2e}
\expandafter\endgroup\ifx\nameoflatex\fmtname\else
\csname fi\endcsname
%</internal>
%<*install>
\input docstrip.tex
\preamble

Copyright (C) 2018-\the\year by Iydon Liang <liangiydon AT gmail.com>

This file may be distributed and/or modified under the
conditions of the LaTeX Project Public License, either version 1.3c
of this license or (at your option) any later version.
The latest version of this license is in
   https://www.latex-project.org/lppl.txt
and version 1.3c or later is part of all distributions of LaTeX
version 2005/12/01 or later.

\endpreamble
\keepsilent
\askforoverwritefalse
\nopostamble
\generate{
  \file{\jobname.cls}{\from{\jobname.dtx}{class}}
}
\endbatchfile
%</install>
%<*internal>
\fi
%</internal>
%<*driver>
\ProvidesFile{sustechthesis.dtx}
%</driver>
%<class>\NeedsTeXFormat{LaTeX2e}[1999/12/01]
%<class>\ProvidesClass{sustechthesis}
%<*class>
  [2019/11/05 v0.1 SUSTech thesis template]
%</class>
%
%<*driver>
\documentclass[a4paper]{ltxdoc}
\usepackage{ctex}
\usepackage{unicode-math}
\usepackage{caption}
\usepackage{booktabs}
\usepackage{xcolor}
\usepackage{listings}
\usepackage{hypdoc}

\makeatletter

% 设置字体

% 定义一些命令用于写文档
\newcommand\TeXLive{\TeX{} Live}
\newcommand\MacTeX{Mac\TeX}
\newcommand\MikTeX{Mik\TeX}
\newcommand\Python{\texttt{Python}}
\newcommand\pip{\texttt{pip}}
\newcommand\unicodechar[1]{U+#1(\symbol{"#1})}
\DeclareRobustCommand\file{\nolinkurl}
\DeclareRobustCommand\env{\texttt}
\DeclareRobustCommand\pkg{\textsf}
\DeclareRobustCommand\cls{\textsf}
\DeclareRobustCommand\opt{\texttt}

% 在 doc 的基础上增加 option 的描述
\def\DescribeOption{\leavevmode\@bsphack\begingroup\MakePrivateLetters
  \Describe@Option}
\def\Describe@Option#1{\endgroup
              \marginpar{\raggedleft\PrintDescribeOption{#1}}%
              \SpecialEnvIndex{#1}\@esphack\ignorespaces}
\@ifundefined{PrintDescribeOption}
   {\def\PrintDescribeOption#1{\strut \MacroFont #1\ }}{}

% 调整列表的格式
\setlength\partopsep{\z@}
\def\@listi{\leftmargin\leftmargini
            \parsep \z@
            \topsep 5\p@ \@plus2\p@ \@minus3\p@
            \itemsep2\p@ \@plus\p@ \@minus\p@}
\let\@listI\@listi
\@listi

% listings 的样式
\lstdefinestyle{lstshell}{
  basicstyle      = \small\ttfamily,
  backgroundcolor = \color{lightgray},
  gobble          = 2, % 重要!否则会生成注释符号"%"
  language        = bash,
}
\newcommand\shellcmd[1]{\colorbox{lightgray}{\lstinline[style=lstshell]|#1|}}
\lstnewenvironment{shell}{\lstset{style=lstshell}}{}
\lstnewenvironment{latex}{%
  \lstset{
    basicstyle = \small\ttfamily,
    frame      = single,
    gobble     = 4,
    language   = [LaTeX]TeX,
  }%
}{}

\hypersetup{
  allcolors         = blue,
  bookmarksnumbered = true,
  bookmarksopen     = true,
}

\makeatother

\EnableCrossrefs
% \OnlyDescription

\begin{document}
  \DocInput{\jobname.dtx}
  \clearpage
\end{document}
%</driver>
% \fi
%
% \DoNotIndex{\def,\long,\edef,\xdef,\gdef,\let,\global}
% \DoNotIndex{\if,\ifnum,\ifdim,\ifcat,\ifmmode,\ifvmode,\ifhmode,%
%             \iftrue,\iffalse,\ifvoid,\ifx,\ifeof,\ifcase,\else,\or,\fi}
% \DoNotIndex{\begin,\end,\bgroup,\egroup,\begingroup,\endgroup}
% \DoNotIndex{\expandafter,\csname,\endcsname}
% \DoNotIndex{\hsize,\vsize,\hskip,\vskip,\kern,\hfil,\hfill,\hss}
% \DoNotIndex{\hspace,\vspace}
% \DoNotIndex{\p@,\m@ne,\z@,\@ne,\tw@,\@plus,\@minus}
% \DoNotIndex{\newcounter,\setcounter,\addtocounter,}
% \DoNotIndex{\newdim,\newlength,\setlength,\addtolength}
% \DoNotIndex{\newcommand,\renewcommand,\providecommand,\DeclareRobustCommand}
% \DoNotIndex{\newenvironment,\renewenvironment}
% \DoNotIndex{\RequirePackage,\LoadClass,\ProvidesClass}
% \DoNotIndex{\DeclareOption,\CurrentOption,\ExecuteOptions,\ProcessOptions}
% \DoNotIndex{\rmfamily,\sffamily,\ttfamily,\bfseries,\mdseries,\itshape,%
%             \textrm,\textsf,\texttt,\textbf,\textmd,\textit,\textsl,\textsc}
% \DoNotIndex{\iint,\iiint,\iiiint,\oint,\oiint,\oiiint,%
%             \intclockwise,\varointclockwise,\ointctrclockwise,\sumint,%
%             \intbar,\intBar,\fint,\cirfnint,\awint,\rppolint,%
%             \scpolint,\npolint,\pointint,\sqint,\intlarhk,\intx,%
%             \intcap,\intcup,\upint,\lowint}
% \DoNotIndex{\a,\b,\c,\d,\e,\f,\g,\h,\i,\j,\k,\l,%
%             \m,\n,\o,\p,\q,\r,\s,\t,\u,\v,\w,\x,\y,\z,%
%             \A,\B,\C,\D,\E,\F,\G,\H,\I,\J,\K,\L,%
%             \M,\N,\O,\P,\Q,\R,\S,\T,\U,\V,\W,\X,\Y,\Z,%
%             \do\#,\$,\%,\&,\@,\\,\{,\},\^,\_,\~,\ ,\,,\!,\',\",\/,\*,\-}
% \DoNotIndex{\NAT@@close,\NAT@@open,\NAT@cite,\NAT@citenum,\NAT@citesuper,%
%             \NAT@citex,\NAT@citexnum,\NAT@cmt,\NAT@ctype,\NAT@date,%
%             \NAT@last@yr,\NAT@mbox,\NAT@penalty,\NAT@spacechar,%
%             \@citea,\def@NAT@last@yr,\ifNAT@swa}
% \DoNotIndex{\quad,\par,\relax,\ccwd}
% \DoNotIndex{\bp@}
%
%
%
% \GetFileInfo{\jobname.dtx}
%
% \title{\cls{sustechthesis} 使用说明\footnote{大部分代码来自 \cls{sustechthesis}}}
% \author{Iydon Liang\thanks{liangiydon AT gmail.com}}
% \date{\filedate\qquad\fileversion}
% \maketitle\tableofcontents\clearpage
%
%
%
% \section{简介}
%
% 本模板 \cls{sustechthesis} 是南方科技大学本科生学位论文的 \LaTeX{} 模板, 按照
% 《\href{http://tao.sustech.edu.cn/studentService/graduation_project.html}
% {南方科技大学本科生毕业论文工作的若干规定(2017年修订)}》的要求编写.
% 其前身是 \href{https://github.com/ustctug/ustcthesis}
% {中国科学技术大学学位论文 \LaTeX{} 模板} (主要维护 seisman 和 zepinglee)
% 和南方科技大学本科论文 \TeX{} 格式模板 (作者 Iydon).
% 后来参考前者对后者进行修订, 由 Iydon 维护, 但是学校官网未发布研究生 学位论文的要求,
% 所以仅完成本科生学位论文的 \LaTeX{} 模板.
% 2019 年, 将原先的 \TeX{} 格式模板转化为文类形式, 所以本模板版本 v0.1.
%
% 下载地址:
% \begin{itemize}
%   \item 主要地址:\url{https://github.com/Iydon/sustechthesis}
% \end{itemize}
%
% 用户在使用 \pkg{sustechthesis} 模板前, 应先阅读学校的《撰写规范》等规范.
% 如果在使用的过程中遇到问题, 可以提出
% \href{https://github.com/Iydon/sustechthesis/issues}{GitHub Issues},
% 预计最后会收集在 \href{https://github.com/ustctug/ustcthesis/wiki}{常见问题} 中.
%
%
%
% \section{编译方法}
% 本节主要取自 \href{https://github.com/ustctug/ustcthesis}{\pkg{ustcthesis}},
% 等待模板制作成功后便对文档进行进一步修改.
%
%
% \subsection{文件组成}
% 本模板的主要文件如表~\ref{tab:files} (todo):
% \begin{table}[htb]
%   \centering\small
%   \caption{模板的文件组成}
%   \label{tab:files}
%   \begin{tabular}{lll}
%     \toprule
%     类别 & 文件 & 说明 \\
%     \midrule
%     模板文件 & \file{sustechthesis.dtx} & 模板原始代码文件, 用户无需使用 \\
%       & \file{sustechthesis.cls} & 文档类文件 \\
%       & \file{figures/logo/SUSTech-*.pdf} & 校名和校徽图片 \\
%     \midrule
%     生成文件 & \file{sustechthesis.pdf} & (你正在阅读的) 模板使用说明 \\
%     \midrule
%     示例文档 & \file{main.tex} & 主文档 \\
%       & \file{chapters/*.tex} & 示例文档的各个章节 \\
%       & \file{figures/} & 放置图片的目录 \\
%       & \file{bib/sustech.bib} & \BibTeX{} 示例数据库 \\
%     \midrule
%     其他 & \file{README.md} & 基本说明 \\
%       & \file{latexmkrc} & latexmk 的配置文件 \\
%     \bottomrule
%   \end{tabular}
% \end{table}
%
% 示例文档包括了常用的 \LaTeX{} 命令, 建议新手看完 \href{https://tex.readthedocs.io/}
% {Read the Docs} 后从此入手, 用自己的内容进行替换.
%
% 文件 \file{sustechthesis.dtx} 是模板的原始代码文件, 可以编译生成文档类文件
% \file{sustechthesis.cls} 和模板使用说明文件 \file{sustechthesis.pdf}.
% 原始模板文件仅供模板开发者使用,一般用户无需使用。
%
%
% \subsection{依赖宏包}
%
% 本模板要求使用 \TeXLive, \MacTeX{} 或 \MikTeX{} 不低于 2017 年的发行版, 推荐升级到最新的版本.
% 如果电脑条件不允许, 建议使用在线编辑器, 如 \href{https://www.overleaf.com/}{Overleaf},
% 如果网络条件不允许, 请联系 Iydon 获得校内在线编辑器的权限, 校内在线编辑器使用的 \TeXLive{}
% 将会定期保持更新.
%
% 模板直接依赖的宏包有 (todo):
% \pkg{amsmath},
% \pkg{caption},
% \pkg{calc},
% \pkg{color},
% \pkg{ctex},
% \pkg{fancyhdr},
% \pkg{footmisc},
% \pkg{geometry},
% \pkg{graphicx},
% \pkg{natbib},
% \pkg{notoccite},
% \pkg{titletoc},
% \pkg{url},
% \pkg{unicode-math}。
%
% 另外, 模板还对其他宏包提供了支持, 包括 (todo):
% \pkg{amsthm},
% \pkg{algorithm2e},
% \pkg{hyperref},
% \pkg{nomencl},
% \pkg{siunitx}.
% 这些宏包并非必需, 用户可以根据需要选择使用.
% 模板在检测到这些宏包被调用后会自动进行配置.
%
% 注意, 本模板\emph{不}兼容的宏包有 (todo):
% \pkg{amsfonts},
% \pkg{amssymb},
% \pkg{biblatex},
% \pkg{bm},
% \pkg{cite},
% \pkg{mathrsfs},
% \pkg{newtx},
% \pkg{upgreek}.
%
%
% \subsection{开始编译}
%
% \begin{enumerate}
%
% \item \Python \\
% 可以直接使用 \pip 工具安装 \pkg{sustechthesis} 库, 这是最简单的方法.
% \begin{shell}
%   pip install sustechthesis
% \end{shell}
% 新建示例文件 \file{main.tex} 及相关目录:
% \begin{shell}
%   sustechthesis new
% \end{shell}
% 编译论文 \file{main.pdf}:
% \begin{shell}
%   sustechthesis build
% \end{shell}
% 查看文档 \file{sustechthesis.pdf}:
% \begin{shell}
%   sustechthesis doc
% \end{shell}
% 另外还可以用 \shellcmd{sustechthesis clean} 清理辅助文件.
%
% \item |latexmk| \\
% 用户可能没有安装 \pip,使用 |latexmk| 也是一个比较简单的方法,
% 配置文件由 \file{latexmkrc} 给出, 其参数设置为 |-xelatex|, 用户编译论文
% 只需使用命令:
% \begin{shell}
%   latexmk -xelatex main.tex
% \end{shell}
% 编译说明文档:
% \begin{shell}
%   latexmk -xelatex sustechthesis.dtx
% \end{shell}
% 清理辅助文件可以用 \shellcmd{latexmk -c}. 图形界面用户应参考编辑器的使用说明.
%
% \item 手动编译 \\
% 手动编译是最繁琐的方法, 用户可能需要运行多遍, 以确保论文的交叉引用等信息全部正确.
%
% 编译论文 \file{main.pdf}:
% \begin{shell}
%   xelatex main
%   bibtex main # 如果不使用 BibTeX 可以略过此步
%   xelatex main
%   xelatex main
% \end{shell}
% 编译说明文档 \file{sustechthesis.pdf}:
% \begin{shell}
%   xelatex sustechthesis.dtx
%   makeindex -s gind.ist sustechthesis.idx
%   makeindex -s gglo.ist -o sustechthesis.gls sustechthesis.glo
%   xelatex sustechthesis.dtx
%   xelatex sustechthesis.dtx
% \end{shell}
% \end{enumerate}
%
%
%
% \section{模板设置}
%
%
% \subsection{文档类参数 (todo)}
% 模板提供了若干选项, 应在载入模板时设置, 如:
% \begin{latex}
%   \documentclass[doctor,english,pdf]{sustechthesis}
% \end{latex}
%
% 全部的选项见表~\ref{tab:options}.
% \begin{table}[htb]
%   \centering\small
%   \caption{文档类参数}
%   \label{tab:options}
%   \begin{tabular}{ll}
%     \toprule
%     文档类参数 & 说明 \\
%     \midrule
%     \opt{doctor} & 博士论文 (todo) \\
%     \opt{master} & 硕士论文 (todo) \\
%     \opt{bachelor}* & 本科论文 \\
%     \opt{academic}* & 学术型学位 \\
%     \opt{professional} & 专业型学位 (限研究生) (todo) \\
%     \opt{chinese}* & 中文 \\
%     \opt{english} & English \\
%     \opt{print}* & 用于双面打印纸质论文 \\
%     \opt{pdf} & 单面打印,并保留超链接颜色 \\
%     \opt{super}* & 角标数字式文献引用标注 \\
%     \opt{numbers} & 数字式文献引用标注 \\
%     \opt{authoryear} & 著者-出版年制文献引用标注 \\
%     \opt{arabic} & 阿拉伯数字式章节标题(限本科生)\\
%     \bottomrule
%   \end{tabular}
%   \caption*{\small 注: 带 * 号的是默认参数}
% \end{table}
%
%
% \subsection{字体设置}
% 本模板是以 \pkg{XeTeX} + \pkg{fontspec} + \pkg{xeCJK}
% 的方式来配置字体的,
% 所以用户必须使用 UTF-8 编码保存源文件, 并且用 |xelatex| 命令进行编译.
%
% \DescribeOption{fontset=\meta{font}}
% 默认情况下, 本模板可以自动检测操作系统, 并配置合适的字体.
% 用户可以在调用文档类时加入选项
% \opt{fontset=mac/windows/adobe} 指定加载的字库,
% 也可以使用 \opt{fontset=none}, 然后自行配置,
% 详见 \pkg{ctex}, \pkg{xeCJK}, \pkg{fontspec} 等宏包的使用说明.
%
% 注意,Linux 系统下可能缺失 Times New Roman 等西文字体,
% 而且默认的中文字库 Fandol 也容易出现缺字的情况.
% 我们建议 Linux 用户自行配置合适的字体.
%
%
%
% \section{论文内容 (todo)}
%
% 本科生论文的内容按照以下顺序排列:
% \begin{description}
%   \item[title page] 中文封面, 英文封面
%   \item[front matter] 致谢, 目录, 中英文摘要
%   \item[main matter] 正文章节, 参考文献
%   \item[appendix] 附录
% \end{description}
%
% 示例文档 \file{main.tex} 中的致谢, 目录等章节的顺序, 是按照本科生论文的格式组织内容的.
%
%
% \subsection{封面}
%
% ``封面''的名字让人有些混淆,它既可以指由印刷厂统一制作的硬皮封面 (cover), 也可
% 以指书打开后的第一页 (title page). 在这里指的是后者, 所以本模板从 title page 开始.
%
% \DescribeMacro{\maketitle}
% \DescribeMacro{\makestatement}
% 封面和声明页分别由 \cs{maketitle} 和 \cs{makestatement} 命令生成,
% 其中的各项信息使用 \cs{\meta{item}\marg{info}} 的方式填写,
% 本模板提供的命令如表~\ref{tab:covercmds},
% 其中带 |en| 前缀的命令是设置英文封面的命令:
% \begin{table}[htb]
%   \centering\small
%   \caption{录入封面信息的命令}
%   \label{tab:covercmds}
%   \begin{tabular}{lll}
%     \toprule
%     命令 & 命令 (英文) & 说明 \\
%     \midrule
%     \cs{title} & \cs{entitle} & 论文标题 \\
%     \cs{author} & \cs{enauthor} & 作者姓名 \\
%     \cs{major} & \cs{enmajor} & 学科专业 \\
%     \cs{supervisor} & \cs{ensupervisor} & 导师姓名 \\
%     \cs{cosupervisor} & \cs{encosupervisor} & 副导师姓名 \\
%     \cs{date} & \cs{endate} & 完成时间 (默认为今天) \\
%     \cs{secretlevel} & \cs{ensecretlevel} & 密级 (默认不保密) \\
%     \cs{secretyear} & - & 保密年限 \\
%     \bottomrule
%   \end{tabular}
% \end{table}
%
%
% \subsection{摘要和章节}
% \DescribeEnv{abstract}
% \DescribeEnv{enabstract}
% \DescribeEnv{notation}
% \DescribeEnv{acknowledgments}
% \DescribeEnv{publications}
% 对于特殊的章节, \cls{sustechthesis} 还提供了相应的环境:
% \begin{itemize}
%   \item 中文摘要: \env{abstract}
%   \item 英文摘要: \env{enabstract}
%   \item 符号说明: \env{notation}
%   \item 致谢: \env{acknowledgments}
%   \item 发表成果: \env{publications}
% \end{itemize}
%
% \DescribeMacro{\keywords}
% \DescribeMacro{\enkeywords}
% 摘要的关键词应使用 \cs{keywords} 和 \cs{enkeywords} 命令, 并包含在摘要环境中, 比如:
% \begin{latex}
%   \begin{abstract}
%     这里是摘要。
%     \keywords{论文;摘要;关键词}
%   \end{abstract}
% \end{latex}
%
% \DescribeMacro{\tableofcontents}
% \DescribeMacro{\listoffigures}
% \DescribeMacro{\listoftables}
% 目录和图, 表清单可以使用命令自动生成:
% \begin{itemize}
%   \item 目录: \cs{tableofcontents}
%   \item 图清单: \cs{listoffigures}
%   \item 表清单: \cs{listoftables}
% \end{itemize}
%
%
% \subsection{浮动体}
%
% 《撰写规范》要求表题置于表的上方 (未说明但作者建议图题置于图的下方).
% \LaTeX{} 的 \cs{caption} 命令并不能控制在浮动体中的位置,
% 需要作者注意写在合适的地方.
%
% \DescribeMacro{\note}
% 本模板还提供了 \cs{note}\marg{notes} 命令, 用于在图表中添加注释.
%
% 关于图片的并排, 推荐使用较新的 \pkg{subcaption} 宏包, 不建议使用
% \pkg{subfigure} 或 \pkg{subfig}.
%
% 更多的表格样式参见 \pkg{booktabs} (三线表), \pkg{longtable} (跨页表格).
%
% 算法可以使用 \pkg{algorithms} 宏包或者较新的 \pkg{algorithm2e}.
%
%
% \subsection{数学}
%
% \LaTeX{} 默认按照美国的习惯排版数学公式和符号,本模板基于 \pkg{unicode-math}
% 宏包配置数学符号, 以遵循国标的规定.
%
% 注意, \pkg{unicode-math} 宏包与 \pkg{amsfonts}, \pkg{amssymb}, \pkg{bm},
% \pkg{mathrsfs}, \pkg{upgreek} 等宏包\emph{不}兼容.
% 本模板作了处理, 用户可以直接使用这些宏包的命令, 如 \cs{bm}, \cs{mathscr},
% \cs{upGamma}.
%
% \DescribeMacro{\symup}
% \DescribeMacro{\symbf}
% 本模板中数学符号的用法与 \LaTeX{} 传统有些区别:
% \begin{itemize}
%   \item 数学常数和特殊函数使用正体,
%     如圆周率 $\symup{\pi}$, $\symup{\Gamma}$ 函数.
%     应使用 \pkg{unicode-math} 宏包提供的 \cs{symup} 命令转为正体,
%     如 |\symup{\pi}|.
%   \item 向量, 矩阵和张量要求粗斜体, 应使用 \cs{symbf} 命令,
%     如 |\symbf{A}|, |\symbf{u}|。
%   \item 有限增量符号 $\increment$ (U+2206)应使用 \cs{increment} 命令.
%   \item 微分符号 $\mathrm{d}$ 使用正体, 本模板提供了 \cs{dif} 命令.
% \end{itemize}
%
% 关于数学符号更多的用法, 参见 \pkg{unicode-math} 宏包的使用说明和符号列表
% \pkg{unimath-symbols}.
%
%
% \subsection{参考文献 (todo)}
%
% \DescribeOption{authoryear}
% \DescribeOption{super}
% \DescribeOption{numbers}
% 按照《撰写规范》和 《GB/T 7714-2015》 的规定,
% 参考文献的标注体系分为``顺序编码制''和``著者-出版年制'' (authoryear),
% 其中顺序编码制根据引用标注样式的不同分为角标数字式 (super)
% 和与正文平排的数字式 (numbers).
%
% 《撰写规范》推荐使用角标数字式顺序编码制 (默认),
% 用户也可以在文档类参数中设置, 如:
% \begin{latex}
%   \documentclass[doctor,authoryear]{ustcthesis}
% \end{latex}
%
% \DescribeMacro{\cite}
% 在正文中引用文献时应使用 \cs{cite} 命令.
% 同一处引用多篇文献时, 需要将各篇文献的 key 一同写在参数中,
% 如 |\cite{knuth84,lamport94,mittelbach04}|.
% 它可以自动排序并用处理连续编号.
% 更多的引用标注方法可以参考 \pkg{natbib} 宏包的使用说明.
%
% \DescribeMacro{\inlinecite}
% 在使用角标数字式时, 如果文献序号作为叙述文字的一部分, 
% 需要临时将文献序号与正文平排, 可以使用 \cs{inlinecite} 命令, 如:
% \begin{latex}
%   文献~\inlinecite{knuth84} 提出了一种新的断行算法
% \end{latex}
%
% 若需要标出引文的页码, 可以标在 \cs{cite} 的可选参数中, 如 |\cite[42]{abc}|.
%
% \DescribeMacro{\citep}
% \DescribeMacro{\citet}
% 在采用著者-出版年制时, 如果将著者姓氏和出版年都置于``( )''内,
% 应使用 \cs{citep} 命令代替 \cs{cite} 来引用;
% 如果将著者姓氏作为正文的一部分, 在其后的``( )''内只著录出版年,
% 则应使用 \cs{citet} 命令.
%
% \DescribeMacro{\bibliography}
% 参考文献表可以使用 \BibTeX{} 生成, 并在文中使用 \cs{bibliography} 命令调用.
% 注意文献列表的样式已经在模板中根据选项设置,
% 用户\emph{不再}需要使用 \cs{bibliographystyle} 命令.
%
% \BibTeX{} 默认情况下可以自动识别文献语言, 并自动处理文献类型和载体类型标识,
% 但是在少数情况下需要用户手动指定, 如:
% \begin{latex}
%   @misc{citekey,
%     language = {japanese},
%     mark     = {Z},
%     medium   = {DK},
%     ...
%   }
% \end{latex}
% 可选的语言有 english, chinese, japanese, russian.
%
% 注意, 国标规定参考文献表采用著者-出版年制组织时, 各篇文献首先按文种集中,
% 然后按著者字顺和出版年排列;
% 中文文献可以按著者汉语拼音字顺排列, 也可以按著者的笔画笔顺排列.
% 然而由于 \BibTeX{} 功能的局限性, 无法自动获取著者姓名的拼音或笔画笔顺,
% 所以\emph{必须}在 bib 数据库中的 key 域手动录入著者姓名的拼音, 如:
% \begin{latex}
%   @book{capital,
%     author = {马克思 and 恩格斯},
%     key    = {ma3 ke4 si1   en1 ge2 si1},
%     ...
%   }
% \end{latex}
%
%
% \StopEventually
% \clearpage
% \appendix
%
%
% \section{代码实现}
% \label{sec:code}
% \linespread{1}
%
%
% \subsection{处理选项}
%
% 检查编译引擎, 要求使用 XeLaTeX.
%    \begin{macrocode}
\RequirePackage{ifxetex}
\RequireXeTeX
%    \end{macrocode}
%
%    \begin{macrocode}
\newif\ifsustech@doctor
\newif\ifsustech@master
\newif\ifsustech@bachelor
\newif\ifsustech@academic
\newif\ifsustech@chinese
\newif\ifsustech@numerical
\newif\ifsustech@super
\newif\ifsustech@pdf
\newif\ifsustech@arabic
\DeclareOption{doctor}{\sustech@doctortrue\sustech@masterfalse\sustech@bachelorfalse}
\DeclareOption{master}{\sustech@doctorfalse\sustech@mastertrue\sustech@bachelorfalse}
\DeclareOption{bachelor}{\sustech@doctorfalse\sustech@masterfalse\sustech@bachelortrue}
\DeclareOption{academic}{\sustech@academictrue}
\DeclareOption{professional}{\sustech@academicfalse}
\DeclareOption{chinese}{\sustech@chinesetrue}
\DeclareOption{english}{\sustech@chinesefalse\sustech@arabictrue}
\DeclareOption{print}{\sustech@pdffalse}
\DeclareOption{pdf}{\sustech@pdftrue}
\DeclareOption{super}{\sustech@numericaltrue\sustech@supertrue}
\DeclareOption{numbers}{\sustech@numericaltrue\sustech@superfalse}
\DeclareOption{authoryear}{\sustech@numericalfalse}
\DeclareOption{arabic}{\sustech@arabictrue}
\PassOptionsToClass{openany}{ctexbook}
\DeclareOption*{\PassOptionsToClass{\CurrentOption}{ctexbook}}
\ExecuteOptions{bachelor, academic, chinese, print, super}
\ProcessOptions\relax
\newif\ifsustech@graduate
\ifsustech@bachelor
  \sustech@graduatefalse
\else
  \sustech@graduatetrue
\fi
%    \end{macrocode}
%
%
% \subsection{加载文档类和宏包}
%
%    \begin{macrocode}
\ifsustech@chinese
  \PassOptionsToClass{scheme=chinese}{ctexbook}
\else
  \PassOptionsToClass{scheme=plain}{ctexbook}
\fi
\ifsustech@pdf
  \PassOptionsToClass{oneside}{book}
\fi
\PassOptionsToPackage{quiet}{xeCJK}
%    \end{macrocode}
%
% 载入 \cls{ctexbook} 文档类, 注意要求为 2.2 或更高的版本.
%    \begin{macrocode}
\LoadClass[UTF8,a4paper,zihao=-4]{ctexbook}[2017/04/01]
%    \end{macrocode}
%
% 检测 ctexbook 版本, 如果版本过低会报错
%    \begin{macrocode}
\@ifclasslater{ctexbook}{2017/04/01}{}{
  \ClassError{sustechthesis}{%
    This template requires TeX Live\MessageBreak 2017 or later version}{}
}
%    \end{macrocode}
%
% 建议在模板开始处载入全部宏包, 不要轻易改变加载顺序.
% \pkg{hyperref} 一般在最后加载.
%    \begin{macrocode}
\RequirePackage{amsmath}
\RequirePackage{unicode-math}
\RequirePackage[driver=xetex]{geometry}
\RequirePackage{graphicx}
\RequirePackage{fancyhdr}
\RequirePackage{color}
\RequirePackage{titletoc}
\RequirePackage{caption}
\RequirePackage[perpage]{footmisc}
\RequirePackage{url}
\RequirePackage{calc}
\ifsustech@numerical
  \PassOptionsToPackage{sort&compress}{natbib}
\fi
\RequirePackage{natbib}
\RequirePackage{notoccite}
%    \end{macrocode}
%
%
% \subsection{字体}
%
% \begin{macro}{\sustech@strifeq}
% 使用 \LaTeX3 的功能判断字符串是否相等. 这里也可以使用 \pkg{xstring} 宏包.
%    \begin{macrocode}
\newcommand\sustech@strifeq{\csname str_if_eq_x:nnTF\endcsname}
%    \end{macrocode}
% \end{macro}
%
% \begin{macro}{\sustech@fontset}
% 将 \pkg{ctex} 的 \opt{fontset} 存在 \cs{sustech@fontset} 方便 \LaTeXe{} 调用.
%    \begin{macrocode}
\newcommand\sustech@fontset{\csname g__ctex_fontset_tl\endcsname}
%    \end{macrocode}
% \end{macro}
%
% 《撰写规范》要求西文字体使用 Times New Roman 和 Arial,
% 但是在 Linux 下没有这两个字体, 所以使用它们的克隆版 TeX Gyre Termes 和
% TeX Gyre Heros.
%    \begin{macrocode}
\sustech@strifeq{\sustech@fontset}{fandol}{
  \setmainfont[
    Extension      = .otf,
    UprightFont    = *-regular,
    BoldFont       = *-bold,
    ItalicFont     = *-italic,
    BoldItalicFont = *-bolditalic,
  ]{texgyretermes}
  \setsansfont[
    Extension      = .otf,
    UprightFont    = *-regular,
    BoldFont       = *-bold,
    ItalicFont     = *-italic,
    BoldItalicFont = *-bolditalic,
  ]{texgyreheros}
  \setmonofont[
    Extension      = .otf,
    UprightFont    = *-regular,
    BoldFont       = *-bold,
    ItalicFont     = *-italic,
    BoldItalicFont = *-bolditalic,
    Scale          = MatchLowercase,
  ]{texgyrecursor}
  \ClassWarningNoLine{sustechthesis}{%
    You are using Fandol font family.\MessageBreak
    Some glyphs may be missing.\MessageBreak
    Please switch to a high-quality font set
  }
}{
  \setmainfont{Times New Roman}
  \setsansfont{Arial}
  \sustech@strifeq{\sustech@fontset}{mac}{
    \setmonofont[Scale=MatchLowercase]{Menlo}
  }{
    \setmonofont[Scale=MatchLowercase]{Courier New}
  }
}
%    \end{macrocode}
%
% 中文字体可以由 \pkg{ctex} 自动设置, 但是会有问题:
% \begin{enumerate}
%   \item 无衬线字体默认会使用微软雅黑或者苹方, 这对打印不太友好;
%   \item 没有粗体的字体不会开启伪粗;
% \end{enumerate}
% 所以需要重新配置一部分, 参考 \pkg{ctex} 宏包.
%    \begin{macrocode}
\sustech@strifeq{\sustech@fontset}{mac}{
  \setCJKmainfont[
       UprightFont = * Light,
          BoldFont = * Bold,
        ItalicFont = Kaiti SC,
    BoldItalicFont = Kaiti SC Bold,
  ]{Songti SC}
  \setCJKsansfont[BoldFont=* Medium]{Heiti SC}
  \setCJKfamilyfont{zhsong}[
       UprightFont = * Light,
          BoldFont = * Bold,
  ]{Songti SC}
  \setCJKfamilyfont{zhhei}[BoldFont=* Medium]{Heiti SC}
  \setCJKfamilyfont{zhkai}[BoldFont=* Bold]{Kaiti SC}
  \xeCJKsetwidth{‘’“”}{1em}
}{
  \xeCJKsetup{EmboldenFactor=2}
  \sustech@strifeq{\sustech@fontset}{windows}{
    \IfFileExists{C:/bootfont.bin}{
      \setCJKmainfont[AutoFakeBold,ItalicFont=KaiTi_GB2312]{SimSun}
      \setCJKfamilyfont{zhkai}[AutoFakeBold]{KaiTi_GB2312}
    }{
      \setCJKmainfont[AutoFakeBold,ItalicFont=KaiTi]{SimSun}
      \setCJKfamilyfont{zhkai}[AutoFakeBold]{KaiTi}
    }
    \setCJKsansfont[AutoFakeBold]{SimHei}
    \setCJKfamilyfont{zhsong}[AutoFakeBold]{SimSun}
    \setCJKfamilyfont{zhhei}[AutoFakeBold]{SimHei}
  }{
    \sustech@strifeq{\sustech@fontset}{adobe}{
      \setCJKmainfont[
        AutoFakeBold = true,
        ItalicFont   = AdobeKaitiStd-Regular,
      ]{AdobeSongStd-Light}
      \setCJKsansfont[AutoFakeBold]{AdobeHeitiStd-Regular}
      \setCJKfamilyfont{zhsong}[AutoFakeBold]{AdobeSongStd-Light}
      \setCJKfamilyfont{zhhei}[AutoFakeBold]{AdobeHeitiStd-Regular}
      \setCJKfamilyfont{zhkai}[AutoFakeBold]{AdobeKaitiStd-Regular}
    }{}
  }
}
%    \end{macrocode}
%
% 使用 \pkg{unicode-math} 配置数学字体.
%    \begin{macrocode}
\unimathsetup{
  math-style = ISO,
  bold-style = ISO,
  nabla      = upright,
  partial    = upright,
}
%    \end{macrocode}
%
% 使用 XITS Math 作为数学字体.
% 注意 XITS 字体于 2018-10-03 更改了字体的文件名, 所以需要判断.
% 原文件名为 \file{xits-regular.otf}, \file{xits-math.otf} 等,
% 后改为 \file{XITS-Regular.otf}, \file{XITSMath-Regular.otf} 等.
%    \begin{macrocode}
\newif\ifsustech@xits@new
\sustech@xits@newtrue
\IfFontExistsTF{xits-math.otf}{
  \sustech@xits@newfalse
}{}
\ifsustech@xits@new
  \setmathfont[
    Extension    = .otf,
    BoldFont     = XITSMath-Bold,
    StylisticSet = 8,
  ]{XITSMath-Regular}
  \setmathfont[range={cal,bfcal},StylisticSet=1]{XITSMath-Regular.otf}
\else
  \setmathfont[
    Extension    = .otf,
    BoldFont     = *bold,
    StylisticSet = 8,
  ]{xits-math}
  \setmathfont[range={cal,bfcal},StylisticSet=1]{xits-math.otf}
\fi
%    \end{macrocode}
%
% \begin{macro}{\sustech@circlefont}
% 研究生的五级节标题和脚注需要使用带圈数字,
% 但 Times New Roman 没有这些符号的字形, 而华文宋体和中易宋体提供了这些字形,
% 配置在 \cs{sustech@circlefont}.
%    \begin{macrocode}
\sustech@strifeq{\sustech@fontset}{mac}{
  \newfontfamily\sustech@circlefont{Songti SC Light}
}{
  \sustech@strifeq{\sustech@fontset}{windows}{
    \newfontfamily\sustech@circlefont{SimSun}
  }{
    \ifsustech@xits@new
      \newfontfamily\sustech@circlefont{XITS-Regular.otf}
    \else
      \newfontfamily\sustech@circlefont{xits-regular.otf}
    \fi
  }
}
%    \end{macrocode}
% \end{macro}
%
% 目前最广泛使用的印刷的长度单位点 (磅) 通常指 PostScript point
% \footnote{\url{https://en.wikipedia.org/wiki/Point_(typography)}},
% 在 \LaTeX{} 中是 bp, 比默认的 pt 略大.
% 这里保存起来可以节约编译时间.
%    \begin{macrocode}
\newdimen\bp@
\bp@=1bp
%    \end{macrocode}
%
% 下面设置字号. 正文字号12bp, 研究生行距20bp, 本科生行距22bp;
% 其他命令的行距按照相同的的比例设置, 如表~\ref{tab:fontsize}.
% \begin{table}[htb]
%   \centering
%   \caption{标准字体命令与字号、行距的对应}
%   \label{tab:fontsize}
%   \begin{tabular}{lllll}
%     \toprule
%     字体命令 & 字号 & bp & 研究生行距 & 本科生行距 \\
%     \midrule
%     \cs{tiny} & 小六 & 6.5 & 10.83 & 11.92 \\
%     \cs{scriptsize} & 六号 & 7.5 & 12.5 & 13.75 \\
%     \cs{footnotesize} & 小五 & 9 & 15 & 16.5 \\
%     \cs{small} & 五号 & 10.5 & 17.5 & 19.25 \\
%     \cs{normalsize} & 小四 & 12 & 20 & 22 \\
%     \cs{large} & 小三 & 15 & 25 & 27.5 \\
%     \cs{Large} & 小二 & 18 & 30 & 33 \\
%     \cs{LARGE} & 二号 & 22 & 36.67 & 40.33 \\
%     \cs{huge} & 小一 & 24 & 40 & 44 \\
%     \cs{Huge} & 一号 & 26 & 43.33 & 47.67 \\
%     \bottomrule
%   \end{tabular}
% \end{table}
%    \begin{macrocode}
\ifsustech@graduate
  \renewcommand\normalsize{%
    \@setfontsize\normalsize{12\bp@}{20\bp@}%
    \abovedisplayskip 12\bp@ \@plus3\bp@ \@minus7\bp@
    \abovedisplayshortskip \z@ \@plus3\bp@
    \belowdisplayshortskip 6.5\bp@ \@plus3.5\bp@ \@minus3\bp@
    \belowdisplayskip \abovedisplayskip
    \let\@listi\@listI}
  \normalsize
%    \end{macrocode}
%
% 注意第~\ref{sec:list} 节去掉了列表的间距, 所以不再修改 \cs{@listi}.
%    \begin{macrocode}
  \renewcommand\small{%
    \@setfontsize\small{10.5\bp@}{17.5\bp@}%
    \abovedisplayskip 10.5\bp@ \@plus3\bp@ \@minus6\bp@
    \abovedisplayshortskip \z@ \@plus3\bp@
    \belowdisplayshortskip 6.5\bp@ \@plus3.5\bp@ \@minus3\bp@
    \belowdisplayskip \abovedisplayskip
  }
  \renewcommand\footnotesize{%
    \@setfontsize\footnotesize{9\bp@}{15\bp@}%
    \abovedisplayskip 9\bp@ \@plus2\bp@ \@minus5\bp@
    \abovedisplayshortskip \z@ \@plus3\bp@
    \belowdisplayshortskip 6\bp@ \@plus3\bp@ \@minus3\bp@
    \belowdisplayskip \abovedisplayskip
  }
  \renewcommand\scriptsize{\@setfontsize\scriptsize{7.5\bp@}{12.5\bp@}}
  \renewcommand\tiny{\@setfontsize\tiny{6.5\bp@}{10.83\bp@}}
  \renewcommand\large{\@setfontsize\large{15\bp@}{25\bp@}}
  \renewcommand\Large{\@setfontsize\Large{18\bp@}{30\bp@}}
  \renewcommand\LARGE{\@setfontsize\LARGE{22\bp@}{36.67\bp@}}
  \renewcommand\huge{\@setfontsize\huge{24\bp@}{40\bp@}}
  \renewcommand\Huge{\@setfontsize\Huge{26\bp@}{43.33\bp@}}
\else
  \renewcommand\normalsize{%
    \@setfontsize\normalsize{12\bp@}{22\bp@}%
    \abovedisplayskip 12\bp@ \@plus3\bp@ \@minus7\bp@
    \abovedisplayshortskip \z@ \@plus3\bp@
    \belowdisplayshortskip 6.5\bp@ \@plus3.5\bp@ \@minus3\bp@
    \belowdisplayskip \abovedisplayskip
    \let\@listi\@listI}
  \normalsize
  \renewcommand\small{%
    \@setfontsize\small{10.5\bp@}{19.25\bp@}%
    \abovedisplayskip 10.5\bp@ \@plus3\bp@ \@minus6\bp@
    \abovedisplayshortskip \z@ \@plus3\bp@
    \belowdisplayshortskip 6.5\bp@ \@plus3.5\bp@ \@minus3\bp@
    \belowdisplayskip \abovedisplayskip
  }
  \renewcommand\footnotesize{%
    \@setfontsize\footnotesize{9\bp@}{16.5\bp@}%
    \abovedisplayskip 9\bp@ \@plus2\bp@ \@minus5\bp@
    \abovedisplayshortskip \z@ \@plus3\bp@
    \belowdisplayshortskip 6\bp@ \@plus3\bp@ \@minus3\bp@
    \belowdisplayskip \abovedisplayskip
  }
  \renewcommand\scriptsize{\@setfontsize\scriptsize{7.5\bp@}{13.75\bp@}}
  \renewcommand\tiny{\@setfontsize\tiny{6.5\bp@}{11.92\bp@}}
  \renewcommand\large{\@setfontsize\large{15\bp@}{27.5\bp@}}
  \renewcommand\Large{\@setfontsize\Large{18\bp@}{33\bp@}}
  \renewcommand\LARGE{\@setfontsize\LARGE{22\bp@}{40.33\bp@}}
  \renewcommand\huge{\@setfontsize\huge{24\bp@}{44\bp@}}
  \renewcommand\Huge{\@setfontsize\Huge{26\bp@}{47.67\bp@}}
\fi
%    \end{macrocode}
%
% 设置行距的倍数为 1.
%    \begin{macrocode}
\linespread{1}\selectfont
%    \end{macrocode}
%
%
% \subsection{处理语言}
%
% 由于 Unicode 的一些标点符号是中西文混用的:
% \unicodechar{00B7},
% \unicodechar{2013},
% \unicodechar{2014},
% \unicodechar{2018},
% \unicodechar{2019},
% \unicodechar{201C},
% \unicodechar{201D},
% \unicodechar{2025},
% \unicodechar{2026},
% \unicodechar{2E3A}.
% 所以要根据语言设置正确的字体.
% \footnote{\url{https://github.com/CTeX-org/ctex-kit/issues/389}}
% 所以要根据语言设置正确的字体.
%    \begin{macrocode}
\newcommand\sustech@setchinese{%
  \xeCJKResetPunctClass
}
\newcommand\sustech@setenglish{%
  \xeCJKDeclareCharClass{HalfLeft}{"2018, "201C}%
  \xeCJKDeclareCharClass{HalfRight}{%
    "00B7, "2019, "201D, "2013, "2014, "2025, "2026, "2E3A%
  }%
}
\newcommand\sustech@setdefaultlanguage{%
  \ifsustech@chinese
    \sustech@setchinese
  \else
    \sustech@setenglish
  \fi
}
%    \end{macrocode}
%
% 这里设置了中英文的各种名字.
%    \begin{macrocode}
\newcommand\sustech@universityname{南方科技大学}
\newcommand\sustech@enuniversityname{Southern University of Science and Technology}
\ifsustech@chinese
  \renewcommand\listfigurename{插图清单}
  \renewcommand\listtablename{表格清单}
  \newcommand\sustech@acknowledgmentsname{致谢}
  \newcommand\sustech@publicationsname{在读期间发表的学术论文与取得的研究成果}
  \newcommand\sustech@notationname{符号说明}
\else
  \renewcommand\figurename{Fig.}
  \newcommand\sustech@acknowledgmentsname{Acknowledgments}
  \newcommand\sustech@publicationsname{Publications}
  \newcommand\sustech@notationname{Notation}
\fi
%    \end{macrocode}
%
%
% \subsection{纸张和页面}
% 使用 \pkg{geometry} 宏包设置纸张和页面.
%
% 纸张: A4;
%
% 页面设置: 上, 下 2.54 cm; 左, 右 3.17 cm; 页眉 1.5 cm, 页脚 1.75 cm.
%
% 注意这里指的是页眉顶部到纸张顶部的距离为 1.5 cm,
% 所以应该是 2.54cm - 0.8cm - 0.24cm = 1.5cm
%
%    \begin{macrocode}
\geometry{
  paper      = a4paper,
  vmargin    = 2.54cm,
  hmargin    = 3.17cm,
  headheight = 0.75cm,
  headsep    = 0.29cm,
  footskip   = 0.79cm,
}
%    \end{macrocode}
%
% 使用 \pkg{fancy} 需要先调用 |\pagestyle{fancy}| 再修改 \cs{chaptermark} 和
% \cs{sectionmark}.
%    \begin{macrocode}
\pagestyle{fancy}
\let\sectionmark\@gobble
\renewcommand\headrulewidth{0.4\p@}
%    \end{macrocode}
%
% 使用 \pkg{ctex} 的 \cs{ctex\_patch\_cmd:Nnn} 命令.
%    \begin{macrocode}
\newcommand\sustech@patchcmd{\csname ctex_patch_cmd:Nnn\endcsname}
%    \end{macrocode}
%
% 页眉与该部分的章标题相同, 宋体 10.5 磅 (五号) 居中.
% 页码: 宋体 10.5 磅, 页面下脚居中.
%    \begin{macrocode}
\ifsustech@graduate
  \newcommand\sustech@hf@font{\fontsize{10.5\bp@}{12\bp@}\selectfont}
  \newcommand\sustech@header{\leftmark}
  \sustech@patchcmd\chaptermark{\MakeUppercase}{}
  \sustech@patchcmd\chaptermark{#1}{\sustech@spacetitle{#1}}
%    \end{macrocode}
%
% 本科生要求除封面, 扉页外, 每面上部加页眉,
% 用小 5 号字 (9 bp) 标注``南方科技大学'', 居中;
% 从目录页开始在每面底部居中用小五宋体 (9 bp) 连续编页码.
%    \begin{macrocode}
\else
  \newcommand\sustech@hf@font{\fontsize{9\bp@}{12\bp@}\selectfont}
  \newcommand\sustech@header{\sustech@universityname{}本科毕业论文}
  \let\chaptermark\@gobble
\fi
%    \end{macrocode}
%
% 重定义默认的 |plain| page style, 会显示页眉和页脚.
%    \begin{macrocode}
\fancypagestyle{plain}{%
  \fancyhf{}%
  \fancyhead[C]{\sustech@hf@font\sustech@header}%
  \fancyfoot[C]{\sustech@hf@font\thepage}%
}
\pagestyle{plain}
%    \end{macrocode}
%
% |headings| 只有页眉, 没有页脚, 用于研究生的符号说明和本科生的 front matter.
%    \begin{macrocode}
\fancypagestyle{headings}{\fancyfoot{}}
%    \end{macrocode}
%
% 每章第一页默认会设置特殊的 page style, 我们希望它不改变页眉页脚,
% 所以定义一个 |none|.
%    \begin{macrocode}
\def\ps@none{}
\ctexset{chapter/pagestyle=none}
%    \end{macrocode}
%
% \begin{macro}{\cleardoublepage}
% 空白页不加页眉和页码.
%    \begin{macrocode}
\def\cleardoublepage{\clearpage\if@twoside \ifodd\c@page\else
  \hbox{}\thispagestyle{empty}\newpage\if@twocolumn\hbox{}\newpage\fi\fi\fi}
%    \end{macrocode}
% \end{macro}
%
% \begin{macro}{\frontmatter}
% 研究生要求从``中文摘要''开始页码用大写罗马数字,
% 而本科生的 frontmatter (只有致谢) 不编页码, 从目录开始页码用阿拉伯数字.
%    \begin{macrocode}
\renewcommand\frontmatter{%
  \cleardoublepage
  \@mainmatterfalse
  \pagenumbering{Roman}%
  \ifsustech@graduate
    \pagestyle{plain}%
  \else
    \pagestyle{headings}%
  \fi
}
%    \end{macrocode}
% \end{macro}
%
% \begin{macro}{\mainmatter}
% 研究生要求``第 1 章''要另页起,但是本科生要求另面起.
%    \begin{macrocode}
\renewcommand\mainmatter{%
  \ifsustech@graduate
    \cleardoublepage
    \pagenumbering{arabic}%
  \else
    \clearpage
  \fi
  \pagestyle{plain}%
  \@mainmattertrue}
%    \end{macrocode}
% \end{macro}
%
%
% \subsection{封面}
%
% \begin{macro}{\maketitle}
% 定义命令用于录入信息.
%    \begin{macrocode}
\def\sustech@define@term#1#2{%
  \expandafter\gdef\csname #1\endcsname##1{%
    \expandafter\gdef\csname sustech@#1\endcsname{##1}%
  }%
  \csname #1\endcsname{#2}%
}
%    \end{macrocode}
%
% 定义用户接口.
%    \begin{macrocode}
\sustech@define@term{title}{论文题目}
\sustech@define@term{author}{XXX}
\sustech@define@term{major}{XXX}
\sustech@define@term{supervisor}{XXX\quad 教授}
\sustech@define@term{cosupervisor}{}
\sustech@define@term{date}{\zhnumsetup{time=Chinese}\zhtoday}
\sustech@define@term{professionaltype}{专业学位类型}
\sustech@define@term{secretlevel}{}
\sustech@define@term{secretyear}{}
\sustech@define@term{entitle}{Title}
\sustech@define@term{enauthor}{XXX}
\sustech@define@term{enmajor}{XXX}
\sustech@define@term{ensupervisor}{Prof. XXX}
\sustech@define@term{encosupervisor}{}
\sustech@define@term{endate}{\CTEX@todayold}
\sustech@define@term{enprofessionaltype}{Professional degree type}
\sustech@define@term{ensecretlevel}{}
\sustech@define@term{keywords}{}
\sustech@define@term{enkeywords}{}
%    \end{macrocode}
%
% 定义一些常量.
%    \begin{macrocode}
\ifsustech@doctor
  \newcommand\sustech@thesisname{博士学位论文}
  \newcommand\sustech@enthesisname{A dissertation for doctor's degree}
\else
  \ifsustech@master
    \newcommand\sustech@thesisname{硕士学位论文}
    \newcommand\sustech@enthesisname{A dissertation for master's degree}
  \else
    \newcommand\sustech@thesisname{学士学位论文}
    \newcommand\sustech@enthesisname{A dissertation for bachelor's degree}
  \fi
\fi
\ifsustech@academic
  \newcommand\sustech@specialityname{学科专业}
\else
  \ifsustech@doctor
    \renewcommand\sustech@thesisname{专业博士学位论文}
  \else
    \renewcommand\sustech@thesisname{专业硕士学位论文}
  \fi
  \sustech@strifeq{\sustech@fontset}{mac}{
    \setCJKfamilyfont{zhli}{Baoli SC}
    \providecommand\lishu{\CJKfamily{zhli}}
  }{
    \ifx\lishu\relax
      \ClassWarningNoLine{sustechthesis}{%
        The lishu font is required.\MessageBreak
        Please configure it manually
      }
    \fi
    \providecommand\lishu{\sffamily}
  }
  \newcommand\sustech@specialityname{专业领域}
\fi
%    \end{macrocode}
%
% \begin{macro}{\sustech@pdfbookmark}
% 添加 PDF 书签, 在 \pkg{hyperref} 载入后才有效.
%    \begin{macrocode}
\newcommand\sustech@pdfbookmark{\@gobble}
%    \end{macrocode}
% \end{macro}
%
% \begin{environment}{titlepage}
% 重定义 \env{titlepage} 环境, 不修改页码.
%    \begin{macrocode}
\renewenvironment{titlepage}{%
  \cleardoublepage
  \if@twocolumn
    \@restonecoltrue\onecolumn
  \else
    \@restonecolfalse\newpage
  \fi
  \thispagestyle{empty}%
}{%
  \if@restonecol\twocolumn \else \newpage \fi
}
%    \end{macrocode}
% \end{environment}
%
% 中文封面:
% 密级仿宋 14 磅;
% 论文类型黑体 56 磅;
% 论文题目黑体 26 磅加粗居中, 单倍行距;
% 作者姓名宋体 16 磅, 单倍行距;
% 注意这里的``单倍行距''的地方开启了``对齐到网格'', 所以实际行距有所偏差,
% 所以只能使用直尺测量.
%    \begin{macrocode}
\newcommand\sustech@makezhtitle{%
  \sustech@setchinese
  \begin{titlepage}%
    \sustech@pdfbookmark{封面}%
    \centering
    \parbox[t][0.6cm][t]{\textwidth}{%
      \raggedleft\fangsong\fontsize{14\bp@}{14\bp@}\selectfont
      \null\sustech@secretlevel\par}\par
    \vskip 0.5cm%
    \includegraphics[height=1.3cm]{figures/ustc_logo_text.pdf}\par
    \vskip 0.6cm%
    {\sffamily\fontsize{56\bp@}{56\bp@}\selectfont
      \sustech@thesisname\par}%
    \ifsustech@academic
      \vskip 2.0cm%
    \else
      \vskip 0.8cm%
      {\lishu\fontsize{26\bp@}{26\bp@}\selectfont
        (\sustech@professionaltype)\par}%
      \vskip 1.0cm%
    \fi
    \textcolor{ustcblue}{%
      \includegraphics[height=4.1cm]{figures/ustc_logo_fig.pdf}}\par
    \vskip 0.9cm%
    \parbox[t][3.5cm][c]{\textwidth}{%
      \centering\sffamily\bfseries\fontsize{26\bp@}{50\bp@}\selectfont
      \sustech@title\par}\par
    \vskip 0.6cm%
    {\fontsize{16\bp@}{31\bp@}\selectfont
      \begin{tabular}{@{}l@{\hspace{\ccwd}}l@{}}%
        \textsf{作者姓名:} & \sustech@author \\
        \textsf{\sustech@specialityname:} & \sustech@major \\
        \textsf{导师姓名:} & \sustech@supervisor\quad\sustech@cosupervisor \\
        \textsf{完成时间:} & \sustech@date
      \end{tabular}\par}%
  \end{titlepage}%
  \sustech@setdefaultlanguage
}
%    \end{macrocode}
%
% 英文封面的 supervisor 一栏需要判断单复数
%    \begin{macrocode}
\newcommand\sustech@supervisorline{%
  \ifx\sustech@encosupervisor\@empty
    Supervisor: \sustech@ensupervisor
  \else
    Supervisors: \sustech@ensupervisor, \sustech@encosupervisor
  \fi
}
%    \end{macrocode}
%
% 英文封面
%    \begin{macrocode}
\newcommand\sustech@makeentitle{%
  \sustech@setenglish
  \begin{titlepage}%
    \sustech@pdfbookmark{Title page}%
    \centering
    \parbox[t][0.4cm][t]{\textwidth}{%
      \raggedleft\fontsize{14\bp@}{14\bp@}\selectfont
      \null\sustech@ensecretlevel\par}\par
    \vskip 0.5cm%
    {\sffamily\fontsize{20\bp@}{30\bp@}\selectfont
      \sustech@enuniversityname\par}%
    {\sffamily\fontsize{26\bp@}{30\bp@}\selectfont
      \sustech@enthesisname\par}%
    \ifsustech@academic\else
      {\fontsize{16\bp@}{32\bp@}\selectfont
        (\sustech@enprofessionaltype)\par}%
    \fi
    \vskip 2.5cm%
    \textcolor{ustcblue}{%
      \includegraphics[height=4.5cm]{figures/ustc_logo_fig.pdf}}\par
    \vskip 0.5cm%
    \parbox[t][4.5cm][c]{\textwidth}{%
      \centering\sffamily\bfseries\fontsize{26\bp@}{30\bp@}\selectfont
      \sustech@entitle\par}\par
    \vskip 1.6cm%
    {\fontsize{16\bp@}{30\bp@}\selectfont
      \begin{tabular}{@{}l@{}}%
        Author:        \sustech@enauthor \\
        Speciality:    \sustech@enmajor \\
        \sustech@supervisorline \\
        Finished time: \sustech@endate
      \end{tabular}\par}%
  \end{titlepage}%
  \sustech@setdefaultlanguage
}
%    \end{macrocode}
%
% 重新定义 \cs{maketitle}, 调用 \cs{sustech@makezhtitle},, \cs{sustech@makeentitle}
% 分别生成中, 英文封面.
%    \begin{macrocode}
\renewcommand\maketitle{%
  \newgeometry{margin=2.54cm}%
  \pagenumbering{Alph}%
  \pagestyle{empty}%
  \sustech@makezhtitle
  \sustech@makeentitle
  \restoregeometry
  \pagestyle{plain}%
}
%    \end{macrocode}
% \end{macro}
%
%%
% \subsection{原创性声明}
%
% 定义原创性声明
%    \begin{macrocode}
\newcommand\sustech@originality{%
  本人声明所呈交的学位论文,是本人在导师指导下进行研究工作所取得的成果。%
  除已特别加以标注和致谢的地方外,论文中不包含任何他人已经发表或撰写过的%
  研究成果。%
  与我一同工作的同志对本研究所做的贡献均已在论文中作了明确的说明。}
\newcommand\sustech@authorization{%
  作为申请学位的条件之一,学位论文著作权拥有者授权\sustech@universityname{}拥有%
  学位论文的部分使用权,%
  即:学校有权按有关规定向国家有关部门或机构送交论文的复印件和电子版,%
  允许论文被查阅和借阅,可以将学位论文编入《中国学位论文全文数据库》等%
  有关数据库进行检索,可以采用影印、缩印或扫描等复制手段保存、汇编学位论文。%
  本人提交的电子文档的内容和纸质论文的内容相一致。\par
  保密的学位论文在解密后也遵守此规定。}
%    \end{macrocode}
%
% \begin{macro}{\sustech@underline}
% 生成空的下划线
%    \begin{macrocode}
\newcommand\sustech@underline[2][3.2cm]{\underline{\hb@xt@ #1{\hss#2\hss}}}
%    \end{macrocode}
% \end{macro}
%
% \begin{macro}{\sustech@checkbox}
%    \begin{macrocode}
\newcommand\sustech@checkbox{%
  \makebox[\z@][l]{$\square$}%
  \raisebox{-0.2ex}{\hspace{0.1em}$\checkmark$}%
}
%    \end{macrocode}
% \end{macro}
%
% \begin{macro}{\makestatement}
%    \begin{macrocode}
\newcommand\makestatement{%
  \begin{titlepage}%
    \sustech@pdfbookmark{原创性和授权使用声明}%
    \null
    \vskip 0.3cm%
    {\centering\sffamily\fontsize{16\bp@}{32\bp@}\selectfont
      \sustech@universityname{}学位论文原创性声明\par}%
    \vskip 0.7cm%
    \sustech@originality\par
    \vskip 1.3cm%
    作者签名:\sustech@underline{}\hspace{2.7cm}%
    签字日期:\sustech@underline{}\par
    \vskip 1.9cm%
    {\centering\sffamily\fontsize{16\bp@}{32\bp@}\selectfont
      \sustech@universityname{}学位论文授权使用声明\par}%
    \vskip 0.7cm%
    \sustech@authorization\par
    \vskip 0.6cm%
    \ifx\sustech@secretlevel\@empty
      \sustech@checkbox{} 公开\quad
      $\square$ 保密(\sustech@underline[0.85cm]{}年)\par
    \else
      $\square$ 公开\quad
      \sustech@checkbox{} 保密(\sustech@underline[0.8cm]{\sustech@secretyear}年)\par
    \fi
    \vskip 0.5cm%
    作者签名:\sustech@underline{}\hspace{2.7cm}%
    导师签名:\sustech@underline{}\par
    \vskip 0.5cm%
    签字日期:\sustech@underline{}\hspace{2.7cm}%
    签字日期:\sustech@underline{}\par
  \end{titlepage}%
}
\ifsustech@bachelor
  \renewcommand\makestatement{}
\fi
%    \end{macrocode}
% \end{macro}
%
%
% \subsection{章节标题}
%
% 标题最多允许使用五级.
%    \begin{macrocode}
\setcounter{secnumdepth}{5}
%    \end{macrocode}
%
% \begin{macro}{\sustech@spacetitle}
% 研究生规定章题为两字时中间空两字, 三字时空一字, 四字时空半字, 四字以上不空.
% 这里用 \LaTeX3 的 \cs{str\_count:N} 判断字数.
% 注意, \pkg{stringstrings} 宏包会导致范数命令 \verb+\|+ 被修改.
%    \begin{macrocode}
\newcount\sustech@titlelength
\DeclareRobustCommand\sustech@spacetitle[1]{%
  \sustech@titlelength=\csname str_count:N\endcsname{#1}%
  \begingroup
    \ifsustech@graduate
      \ifcase\sustech@titlelength
      \or\or
        \ziju{2}%
      \or
        \ziju{1}%
      \or
        \ziju{0.5}%
      \fi
    \else
      \if@mainmatter\else
        \ifcase\sustech@titlelength
        \or\or
          \ziju{1}%
        \or\or
          \ziju{0.5}%
        \fi
      \fi
    \fi
    #1%
  \endgroup
}
%    \end{macrocode}
% \end{macro}
%
% \begin{macro}{\sustech@textcircled}
% 五级节标题和脚注需要使用带圈的数字, 这里使用 \cs{sustech@circlefont}:
%    \begin{macrocode}
\newcommand\sustech@textcircled[1]{%
  \ifnum\value{#1}<21\relax
    {\sustech@circlefont\symbol{\numexpr\value{#1} + "245F\relax}}%
  \else
    \ClassError{ustcthesis}{Cannot display more than 10 footnotes}{}%
  \fi
}
%    \end{macrocode}
% \end{macro}
%
% 用 \pkg{ctex} 的接口设置全部章节标题格式.
%
% 各章标题: 黑体 16 磅加粗居中, 单倍行距, 段前 24 磅, 段后 18 磅,
% 章序号与章名间空一字.
%
% 由于 Word 模板中使用``单倍行距'', 还``对齐到网格'', 这在 \TeX{} 中不容易实现.
% 所以目前按照默认的行距.
%
% 注意 \pkg{ctex} v2.4.3 以下版本的bug会导致章节标题前后的距离的实际值偏大.
% 另外 \pkg{ctex} v2.2 前的beforeskip的符号有特殊意义.
%    \begin{macrocode}
\ctexset{
  chapter = {
    format      = \centering\sffamily\bfseries\fontsize{16\bp@}{26.67\bp@}\selectfont,
    nameformat  = {},
    titleformat = \sustech@spacetitle,
    number      = \thechapter,
    aftername   = \hspace{\ccwd},
    beforeskip  = 24\bp@,
    afterskip   = 18\bp@,
    fixskip     = true,
  },
}
%    \end{macrocode}
%
% 一级节标题: 黑体 14 磅左顶格, 单倍行距, 段前 24 磅, 段后 6 磅,
% 序号与题名间空一字.
%    \begin{macrocode}
\ctexset{
  section = {
    format     = \sffamily\fontsize{14\bp@}{23.33\bp@}\selectfont,
    aftername  = \hspace{\ccwd},
    beforeskip = 24\bp@,
    afterskip  = 6\bp@,
  },
%    \end{macrocode}
%
% 二级节标题: 黑体 13 磅, 左缩进两字, 单倍行距, 段前 12 磅, 段后 6 磅,
% 序号与题名间空一字.
%    \begin{macrocode}
  subsection = {
    format     = \sffamily\fontsize{13\bp@}{21.67\bp@}\selectfont,
    aftername  = \hspace{\ccwd},
    indent     = 2\ccwd,
    beforeskip = 12\bp@,
    afterskip  = 6\bp@,
  },
%    \end{macrocode}
%
% 三级及以下节标题的格式没有具体规定, 按照 Word 模板的格式:
% 使用黑体 12 磅, 左缩进两字, 行距 20 磅, 段前段后 0 磅, 序号与题名间空半字宽.
%    \begin{macrocode}
  subsubsection = {
    format     = \sffamily\fontsize{12\bp@}{20\bp@}\selectfont,
    number     = \arabic{subsubsection},
    aftername  = .\hspace{0.5\ccwd},
    indent     = 2\ccwd,
    beforeskip = \z@,
    afterskip  = \z@,
  },
%    \end{macrocode}
%
% 按照 Word 模板的格式, 四级节标题: 宋体 12 磅, 左缩进两字, 行距 20 磅,
% 段前段后 0 磅, 序号使用全宽括号, 与题名间空半字宽.
%    \begin{macrocode}
  paragraph = {
    format     = \rmfamily\fontsize{12\bp@}{20\bp@}\selectfont,
    number     = (\arabic{paragraph}),
    aftername  = \hspace{0.5\ccwd},
    indent     = 2\ccwd,
    beforeskip = \z@,
    afterskip  = \z@,
    runin      = false,
  },
%    \end{macrocode}
%
% 按照 Word 模板的格式, 五级节标题: 宋体 12 磅, 左缩进两字, 行距 20 磅,
% 段前段后 0 磅, 序号使用全宽括号, 与题名间空半字宽.
%    \begin{macrocode}
  subparagraph = {
    format     = \rmfamily\fontsize{12\bp@}{20\bp@}\selectfont,
    number     = \sustech@textcircled{subparagraph},
    aftername  = \hspace{0.5\ccwd},
    indent     = 2\ccwd,
    beforeskip = \z@,
    afterskip  = \z@,
    runin      = false,
  },
}
%    \end{macrocode}
%
% 在研究生格式的基础上再设置本科生的章节标题格式.
%    \begin{macrocode}
\ifsustech@bachelor
  \setcounter{secnumdepth}{4}
  \ctexset{
%    \end{macrocode}
%
% 论文的致谢, 目录, 摘要和参考文献等标题用小二号 (18 bp) 黑体字, 居中,
% 这通过 \cs{if@mainmatter} 区分.
% 正文中的标题分章, 节, 段三级; 章, 节标题居中, 段标题居左,
% 分别用三号 (16 bp) 黑体, 小三 (15 bp) 黑体, 四号 (14 bp) 黑体.
%    \begin{macrocode}
    chapter = {
      format = {
        \centering\sffamily
        \if@mainmatter
          \fontsize{16\bp@}{29.33\bp@}\selectfont
        \else
          \fontsize{18\bp@}{33\bp@}\selectfont
        \fi
      },
    },
    section = {
      format = \centering\sffamily\fontsize{15\bp@}{27.5\bp@}\selectfont,
    },
    subsection = {
      format    = \sffamily\fontsize{14\bp@}{25.67\bp@}\selectfont,
      indent    = \z@,
    },
    subsubsection = {
      format    = \rmfamily\fontsize{12\bp@}{22\bp@}\selectfont,
      indent    = \ccwd,
    },
    paragraph = {
      format    = \rmfamily\fontsize{12\bp@}{22\bp@}\selectfont,
      aftername = {},
      indent    = \ccwd,
    },
  }
%    \end{macrocode}
%
% 本科生的阿拉伯数字式标题的格式与研究生几乎一致, 只有中文数字式需要修改.
%    \begin{macrocode}
  \ifsustech@arabic\else
    \ctexset{
      chapter = {
        number = \chinese{chapter},
      },
      section = {
        name   = {第,节},
        number = \chinese{section},
      },
      subsection = {
        number    = \chinese{subsection},
        aftername = {、},
      },
    }
  \fi
\fi
%    \end{macrocode}
%
% \begin{macro}{\sustech@chapter}
% 默认的 \cs{chapter*} 生成的章标题没有编号, 不更改页眉,
% 也不添加进目录或 PDF 书签.
% 然而像摘要, 目录, 符号说明这样的章节, 它们不需要编号, 不加入目录, 
% 但是需要修改页眉, 并且加入 PDF 标签.
% 所以我们新定义 \cs{sustech@chapter} 用于处理这些章节.
%    \begin{macrocode}
\NewDocumentCommand\sustech@chapter{o m}{%
  \if@openright\cleardoublepage\else\clearpage\fi
  \IfValueTF{#1}{%
    \sustech@pdfbookmark{#1}%
    \chaptermark{#1}%
  }{%
    \sustech@pdfbookmark{#2}%
    \chaptermark{#2}%
  }%
  \chapter*{#2}}
%    \end{macrocode}
% \end{macro}
%
%
% \subsection{摘要}
%
% \begin{environment}{abstract}
% 中文摘要环境.
% 判断 \cs{sustech@tocloaded} 是为了防止本科生未调整摘要位置时的目录页码缺失.
%    \begin{macrocode}
\newenvironment{abstract}{%
  \sustech@setchinese
  \ifsustech@graduate
    \sustech@chapter{摘要}%
  \else
    \chapter{中文内容摘要}%
    \ifsustech@tocloaded\else
      本科生需要手动将摘要置于目录后。\par
    \fi
  \fi
}{
  \par\null\par\noindent\hangindent=4\ccwd\relax
  \textbf{关键词}:\sustech@keywords
  \sustech@setdefaultlanguage
}
%    \end{macrocode}
% \end{environment}
%
% \begin{environment}{enabstract}
% 英文摘要环境
%    \begin{macrocode}
\newenvironment{enabstract}{%
  \sustech@setenglish
  \ifsustech@graduate
    \sustech@chapter[Abstract]{ABSTRACT}%
  \else
    \chapter[英文内容摘要]{Abstract}%
  \fi
}{
  \par\null\par\noindent\hangindent=5.3em\relax
  \textbf{Key Words}: \sustech@enkeywords
  \ifsustech@graduate
    \cleardoublepage
  \fi
  \sustech@setdefaultlanguage
}
%    \end{macrocode}
% \end{environment}
%
%
% \subsection{目录}
%
% 判断是否已经加载目录, 用于提醒本科生更改摘要和致谢的顺序.
%    \begin{macrocode}
\newif\ifsustech@tocloaded
%    \end{macrocode}
%
% \begin{macro}{\tableofcontents}
% 研究生规定目录另面起;
% 本科生规定从目录开始编页码, 所以必须另页起.
%    \begin{macrocode}
\renewcommand\tableofcontents{%
  \ifsustech@bachelor
    \cleardoublepage
    \pagenumbering{arabic}%
    \pagestyle{plain}%
    \sustech@tocloadedtrue
  \fi
  \sustech@chapter{\contentsname}%
  \@starttoc{toc}%
}
%    \end{macrocode}
% \end{macro}
%
% 下面用 \pkg{titletoc} 宏包设置目录内容的格式.
% 先定义目录线:
%    \begin{macrocode}
\newcommand\sustech@leaders{\titlerule*[9\bp@]{\textperiodcentered}}
%    \end{macrocode}
%
% 各章目录要求宋体 14 磅, 单倍行距, 段前 6 磅, 段后 0 磅, 两端对齐,
% 页码右对齐, 章序号与章名间空一字.
% 但是 Word 模板中实际是行距 20 磅.
%
% 另外 \pkg{ctex} 在章目录的序号后加 |\hspace{.3em}|, 所以用 \cs{unskip} 修正.
%    \begin{macrocode}
\ifsustech@graduate
  \titlecontents{chapter}
    [\z@]{\addvspace{6\bp@}\fontsize{14\bp@}{20\bp@}\selectfont}
    {\contentspush{\thecontentslabel\unskip\hspace{\ccwd}}}{}
    {\fontsize{12\bp@}{20\bp@}\selectfont\sustech@leaders\contentspage}
%    \end{macrocode}
%
% 一级节标题目录要求宋体 12 磅, 单倍行距, 左缩进一字, 段前 6 磅, 段后 0 磅,
% 两端对齐, 页码右对齐, 序号与题名间空一字.
% Word 模板中实际是行距 20 磅, 段前 0 磅.
%    \begin{macrocode}
  \titlecontents{section}
    [\ccwd]{\fontsize{12\bp@}{20\bp@}\selectfont}
    {\contentspush{\thecontentslabel\hspace{\ccwd}}}{}
    {\fontsize{12\bp@}{20\bp@}\selectfont\sustech@leaders\contentspage}
%    \end{macrocode}
%
% 二级节标题目录要求宋体 10.5 磅, 单倍行距, 左缩进两字, 段前 6 磅, 段后 0 磅,
% 两端对齐, 页码右对齐, 序号与题名间空一字.
% Word 模板中实际是行距 20 磅, 段前 0 磅.
%    \begin{macrocode}
  \titlecontents{subsection}
    [2\ccwd]{\fontsize{10.5\bp@}{20\bp@}\selectfont}
    {\contentspush{\thecontentslabel\hspace{\ccwd}}}{}
    {\fontsize{12\bp@}{20\bp@}\selectfont\sustech@leaders\contentspage}
%    \end{macrocode}
%
% 本科生的目录使用小四宋体 (同正文字体), 其他同研究生的格式相近.
%    \begin{macrocode}
\else
  \titlecontents{chapter}
    [\z@]{\normalsize}
    {\contentspush{\thecontentslabel\unskip\hspace{\ccwd}}}{}
    {\sustech@leaders\contentspage}
  \titlecontents{section}
    [\ccwd]{\normalsize}
    {\contentspush{\thecontentslabel\hspace{\ccwd}}}{}
    {\sustech@leaders\contentspage}
  \titlecontents{subsection}
    [2\ccwd]{\normalsize}
    {\contentspush{\thecontentslabel
      \ifsustech@arabic\hspace{\ccwd}\else 、\fi}}{}
    {\sustech@leaders\contentspage}
\fi
%    \end{macrocode}
%
% 本科生要求目录中正文每章前多空一行, 而目录, 附录等章则不需要空行,
% 所以不能简单判断 \cs{if@mainmatter}, 需要重新定义 \cs{mainmatter} 等命令.
%    \begin{macrocode}
\newif\ifsustech@addtocspace
\ifsustech@bachelor
  \sustech@addtocspacetrue
  \g@addto@macro\frontmatter{\sustech@addtocspacefalse}%
  \g@addto@macro\mainmatter{\sustech@addtocspacetrue}%
  \g@addto@macro\backmatter{\sustech@addtocspacefalse}%
  \g@addto@macro\appendix{\sustech@addtocspacefalse}%
\fi
%    \end{macrocode}
%
% \begin{macro}{\chapter}
% 处理本科生在目录中添加空行.
%    \begin{macrocode}
\renewcommand\chapter{%
  \if@openright\cleardoublepage\else\clearpage\fi
  \thispagestyle{\CTEX@chapter@pagestyle}%
  \global\@topnum\z@
  \@afterindenttrue
  \ifsustech@bachelor
    \ifsustech@addtocspace
      \addtocontents{toc}{\protect\addvspace{12\bp@}}%
    \fi
  \fi
  \secdef\@chapter\@schapter
}
%    \end{macrocode}
% \end{macro}
%
% \begin{macro}{\listoffigures}
% 研究生要求图, 表的清单须另页起.
%    \begin{macrocode}
\renewcommand\listoffigures{%
  \ifsustech@graduate
    \cleardoublepage
  \fi
  \sustech@chapter{\listfigurename}%
  \@starttoc{lof}%
}
\titlecontents{figure}
  [2.3em]{\normalsize}
  {\contentslabel{2.3em}}{}
  {\sustech@leaders\contentspage}
%    \end{macrocode}
% \end{macro}
%
% \begin{macro}{\listoftables}
%    \begin{macrocode}
\renewcommand\listoftables{%
  \ifsustech@graduate
    \cleardoublepage
  \fi
  \sustech@chapter{\listtablename}%
  \@starttoc{lot}%
}
\titlecontents{table}
  [2.3em]{\normalsize}
  {\contentslabel{2.3em}}{}
  {\sustech@leaders\contentspage}
%    \end{macrocode}
% \end{macro}
%
%
% \subsection{符号说明}
%
% \begin{environment}{notation}
% 研究生规定符号说明另页起, 标题字体字号等同论文正文,
% 《撰写规范》第 9 页 1.10(2) 还规定``符号说明不加页码''.
%    \begin{macrocode}
\newenvironment{notation}{%
  \ifsustech@graduate
    \cleardoublepage
    \pagestyle{headings}%
  \fi
  \sustech@chapter{\sustech@notationname}%
}{%
  \ifsustech@graduate
    \clearpage
    \pagestyle{plain}%
  \fi
}%
\newcommand*\notationlabel[1]{#1\hfil}%
\newenvironment{notationlist}[1]{%
  \list{}{%
    \itemsep 3pt%
    \labelwidth #1\relax%
    \labelsep 1em%
    \leftmargin\labelwidth
    \advance\leftmargin\labelsep
    \advance\leftmargin 3em%
    \rightmargin 3em%
    \let\makelabel\notationlabel
  }%
}{%
  \endlist
}
%    \end{macrocode}
% \end{environment}
%
%
% \subsection{正文}
%
% \cs{sloppy} 可以减少``overfull boxes''.
%    \begin{macrocode}
\sloppy
%    \end{macrocode}
%
% 禁止扩大段间距. (\href{https://github.com/ustctug/ustcthesis/issues/209}{
%   ustctug/ustcthesis\#209})
%    \begin{macrocode}
\raggedbottom
%    \end{macrocode}
%
% 段间距 0 磅.
%    \begin{macrocode}
\setlength{\parskip}{\z@}
%    \end{macrocode}
%
% 在 \opt{scheme=plain} 时也首段缩进.
%    \begin{macrocode}
\let\@afterindentfalse\@afterindenttrue
\@afterindenttrue
%    \end{macrocode}
%
% URL 的字体设为保持原样.
%    \begin{macrocode}
\urlstyle{same}
%    \end{macrocode}
%
% 使用 \pkg{xurl} 宏包的方法, 增加 URL 可断行的位置.
%    \begin{macrocode}
\def\UrlBreaks{%
  \do\/%
  \do\a\do\b\do\c\do\d\do\e\do\f\do\g\do\h\do\i\do\j\do\k\do\l%
     \do\m\do\n\do\o\do\p\do\q\do\r\do\s\do\t\do\u\do\v\do\w\do\x\do\y\do\z%
  \do\A\do\B\do\C\do\D\do\E\do\F\do\G\do\H\do\I\do\J\do\K\do\L%
     \do\M\do\N\do\O\do\P\do\Q\do\R\do\S\do\T\do\U\do\V\do\W\do\X\do\Y\do\Z%
  \do0\do1\do2\do3\do4\do5\do6\do7\do8\do9\do=\do/\do.\do:%
  \do\*\do\-\do\~\do\'\do\"\do\-}
\Urlmuskip=0mu plus 0.1mu
%    \end{macrocode}
%
% \begin{macro}{\footnote}
% 脚注用带圈的数字:
%    \begin{macrocode}
\renewcommand\thefootnote{\sustech@textcircled{footnote}}
%    \end{macrocode}
%
% LaTeX 默认脚注按章计数, 即每章的开始才重置脚注计数器; 我们修改为按页计数.
% 简单的|\@addtoreset{footnote}{page}|并不可靠,
% \footnote{\url{https://texfaq.org/FAQ-footnpp.html}}
% 所以我们使用 \pkg{footmisc} 宏包.
%
% 脚注线长为版心宽度的四分之一:
%    \begin{macrocode}
\renewcommand\footnoterule{%
  \kern-3\p@
  \hrule\@width.25\textwidth
  \kern2.6\p@}
%    \end{macrocode}
%
% 注文缩进两字:
%    \begin{macrocode}
\renewcommand\@makefntext[1]{%
  \parindent 2\ccwd\relax
  \noindent
  \hb@xt@2\ccwd{\hss\@makefnmark}#1}
%    \end{macrocode}
% \end{macro}
%
%
% \subsection{列表}
% \label{sec:list}
%
% \begin{environment}{enumerate}
% \begin{environment}{description}
% \begin{environment}{itemize}
% 调整列表中各项之间过大的间距.
%    \begin{macrocode}
\setlength\partopsep{\z@}
\newcommand\sustech@nolistsep{%
  \parsep 0\p@ \@plus.2\p@
  \topsep 0\p@ \@plus.2\p@
  \itemsep0\p@ \@plus.2\p@
}
\def\@listi{\leftmargin\leftmargini
            \sustech@nolistsep}
\let\@listI\@listi
\@listi
\def\@listii {\leftmargin\leftmarginii
              \labelwidth\leftmarginii
              \advance\labelwidth-\labelsep
              \sustech@nolistsep}
\def\@listiii{\leftmargin\leftmarginiii
              \labelwidth\leftmarginiii
              \advance\labelwidth-\labelsep
              \sustech@nolistsep}
%    \end{macrocode}
% \end{environment}
% \end{environment}
% \end{environment}
%
%
% \subsection{浮动体}
%
% \LaTeX{} 对放置浮动体的要求比较强, 这里按照 UK TeX FAQ 的建议
% \footnote{\url{https://texfaq.org/FAQ-floats}} 对其适当放宽.
%    \begin{macrocode}
\renewcommand\topfraction{.85}
\renewcommand\bottomfraction{.7}
\renewcommand\textfraction{.15}
\renewcommand\floatpagefraction{.66}
\renewcommand\dbltopfraction{.66}
\renewcommand\dblfloatpagefraction{.66}
\setcounter{topnumber}{9}
\setcounter{bottomnumber}{9}
\setcounter{totalnumber}{20}
\setcounter{dbltopnumber}{9}
%    \end{macrocode}
%
% 修改默认的浮动体描述符为 |htb|.
%    \begin{macrocode}
\def\fps@figure{htb}
\def\fps@table{htb}
%    \end{macrocode}
%
% 用 \pkg{caption} 宏包设置图, 表的格式:
%
% 图号, 图题置于图的下方, 宋体 10.5 磅居中, 单倍行距, 段前 6 磅, 段后 12 磅,
% 图号与图题文字之间空一字, 图号, 图题加粗.
% 图注位于图的下方, 左缩进两字, 续行悬挂缩进左对齐, 两端对齐.
%
% 表号, 表题置于表的上方, 宋体 10.5 磅居中, 单倍行距, 段前 6 磅, 段后 6 磅,
% 表号与表题文字之间空一字, 表号, 表题加粗.
% 表注左缩进两字, 续行悬挂缩进左对齐, 两端对齐.
%    \begin{macrocode}
\setlength{\floatsep}{6\bp@}
\setlength{\textfloatsep}{6\bp@}
\setlength{\intextsep}{6\bp@}
\DeclareCaptionLabelSeparator{zhspace}{\hspace{\ccwd}}
\captionsetup{
  format         = hang,
  font           = small,
  labelsep       = zhspace,
  skip           = 6\bp@,
  figureposition = bottom,
  tableposition  = top,
}
\ifsustech@graduate
  \captionsetup{font+=bf}
\fi
\captionsetup[figure]{
  belowskip = 6\bp@,
}
%    \end{macrocode}
%
% \begin{macro}{\note}
% 新定义了 \cs{note} 来生成图表的附注.
% 如果用 \cs{caption} 生成图表的附注会导致图表的序号有误;
% 如果用 \cs{bicaption} 会导致表注无法置于表后, 而且对齐方式不对.
%    \begin{macrocode}
\newcommand\note[1]{%
  \begingroup
  \captionsetup{
    format        = plain,
    font          = small,
    justification = justified,
    margin        = 2\ccwd,
    position      = bottom,
  }%
  \caption*{#1}%
  \endgroup
}
%    \end{macrocode}
% \end{macro}
%
%
% \subsection{数学符号}
%
% 根据中文的数学排印习惯进行设置:
%
% \begin{macro}{\ldots}
% 省略号一律居中, 所以 \cs{ldots} 不再居于底部.
%    \begin{macrocode}
\ifsustech@chinese
  \def\mathellipsis{\cdots}
\fi
%    \end{macrocode}
% \end{macro}
%
% \begin{macro}{\le}
% \begin{macro}{\ge}
% 小于等于号, 大于等于号要使用倾斜的字形.
%    \begin{macrocode}
\protected\def\le{\leqslant}
\protected\def\ge{\geqslant}
\AtBeginDocument{%
  \renewcommand\leq{\leqslant}%
  \renewcommand\geq{\geqslant}%
}
%    \end{macrocode}
% \end{macro}
% \end{macro}
%
% \begin{macro}{\int}
% 积分号的上下限默认置于上下两侧.
%    \begin{macrocode}
\removenolimits{%
  \int\iint\iiint\iiiint\oint\oiint\oiiint
  \intclockwise\varointclockwise\ointctrclockwise\sumint
  \intbar\intBar\fint\cirfnint\awint\rppolint
  \scpolint\npolint\pointint\sqint\intlarhk\intx
  \intcap\intcup\upint\lowint
}
%    \end{macrocode}
% \end{macro}
%
% \begin{macro}{\Re}
% \begin{macro}{\Im}
% 实部, 虚部操作符使用罗马体 $\mathrm{Re}$, $\mathrm{Im}$ 而不是 fraktur 体
% $\Re$, $\Im$.
%    \begin{macrocode}
\AtBeginDocument{%
  \renewcommand\Re{\operatorname{Re}}%
  \renewcommand\Im{\operatorname{Im}}%
}
%    \end{macrocode}
% \end{macro}
% \end{macro}
%
% \begin{macro}{\nabla}
% \cs{nabla} 使用粗正体.
%    \begin{macrocode}
\AtBeginDocument{%
  \renewcommand\nabla{\mbfnabla}%
}
%    \end{macrocode}
% \end{macro}
%
% \begin{macro}{\bm}
% \begin{macro}{\boldsymbol}
% 兼容旧的粗体命令: \pkg{bm} 的 \cs{bm} 和 \pkg{amsmath} 的 \cs{boldsymbol}.
%    \begin{macrocode}
\newcommand\bm{\symbf}
\renewcommand\boldsymbol{\symbf}
%    \end{macrocode}
% \end{macro}
% \end{macro}
%
% \begin{macro}{\square}
% 兼容 \pkg{amssymb} 中的命令.
%    \begin{macrocode}
\newcommand\square{\mdlgwhtsquare}
%    \end{macrocode}
% \end{macro}
%
% 提供一些方便的命令:
%    \begin{macrocode}
\newcommand\upe{\symup{e}}
\newcommand\upi{\symup{i}}
\newcommand\dif{\mathop{}\!\mathrm{d}}
\DeclareMathOperator*{\argmax}{arg\,max}
\DeclareMathOperator*{\argmin}{arg\,min}
%    \end{macrocode}
%
%
% \subsection{参考文献}
%
% \begin{macro}{\citestyle}
% 定义接口切换引用文献的标注法, 可用 \cs{citestyle} 调用 \opt{super},
% \opt{authoryear} 或 \opt{numbers}.
%    \begin{macrocode}
\newcommand\bibstyle@super{\bibpunct{[}{]}{,}{s}{,}{\textsuperscript{,}}}
\newcommand\bibstyle@numbers{\bibpunct{[}{]}{,}{n}{,}{,}}
\newcommand\bibstyle@authoryear{\bibpunct{(}{)}{;}{a}{,}{,}}
%    \end{macrocode}
% \end{macro}
%
% 处理宏包选项.
%    \begin{macrocode}
\ifsustech@graduate
  \ifsustech@numerical
    \ifsustech@super
      \RequirePackage[super]{gbt7714}
    \else
      \RequirePackage[numbers]{gbt7714}
    \fi
  \else
    \RequirePackage[authoryear]{gbt7714}
  \fi
\else
  \ifsustech@super
    \RequirePackage[super]{gbt7714}
  \else
    \RequirePackage[numbers]{gbt7714}
  \fi
  \bibliographystyle{ustcthesis-bachelor}
\fi
%    \end{macrocode}
%
% \begin{macro}{\cite}
% 下面修改引用标注的格式, 主要是将页码写在上标位置.
% Numerical 模式的 \cs{citet} 的页码:
%    \begin{macrocode}
\sustech@patchcmd{\NAT@citexnum}{%
  \@ifnum{\NAT@ctype=\z@}{%
    \if*#2*\else\NAT@cmt#2\fi
  }{}%
  \NAT@mbox{\NAT@@close}%
}{%
  \NAT@mbox{\NAT@@close}%
  \@ifnum{\NAT@ctype=\z@}{%
    \if*#2*\else\textsuperscript{#2}\fi
  }{}%
}
%    \end{macrocode}
%
% Numerical 模式的 \cs{citep} 的页码:
%    \begin{macrocode}
\renewcommand\NAT@citesuper[3]{\ifNAT@swa
\if*#2*\else#2\NAT@spacechar\fi
\unskip\kern\p@\textsuperscript{\NAT@@open#1\NAT@@close\if*#3*\else#3\fi}%
   \else #1\fi\endgroup}
\renewcommand\NAT@citenum%
    [3]{\ifNAT@swa\NAT@@open\if*#2*\else#2\NAT@spacechar\fi
        #1\NAT@@close\if*#3*\else\textsuperscript{#3}\fi\else#1\fi\endgroup}
%    \end{macrocode}
%
% Author-year 模式的 \cs{citet} 的页码:
%    \begin{macrocode}
\sustech@patchcmd{\NAT@citex}{%
  \if*#2*\else\NAT@cmt#2\fi
  \if\relax\NAT@date\relax\else\NAT@@close\fi
}{%
  \if\relax\NAT@date\relax\else\NAT@@close\fi
  \if*#2*\else\textsuperscript{#2}\fi
}
%    \end{macrocode}
%
% Author-year 模式的 \cs{citep} 的页码:
%    \begin{macrocode}
\renewcommand\NAT@cite%
    [3]{\ifNAT@swa\NAT@@open\if*#2*\else#2\NAT@spacechar\fi
        #1\NAT@@close\if*#3*\else\textsuperscript{#3}\fi\else#1\fi\endgroup}
%    \end{macrocode}
%
% 在顺序编码制下, \pkg{natbib} 只有在三个以上连续文献引用才会使用连接号,
% 这里修改为允许两个引用使用连接号.
% \footnote{\url{https://tex.stackexchange.com/a/86991/82731}}
%    \begin{macrocode}
\sustech@patchcmd{\NAT@citexnum}{%
  \ifx\NAT@last@yr\relax
    \def@NAT@last@yr{\@citea}%
  \else
    \def@NAT@last@yr{--\NAT@penalty}%
  \fi
}{%
  \def@NAT@last@yr{-\NAT@penalty}%
}
%    \end{macrocode}
% \end{macro}
%
% \begin{macro}{\inlinecite}
% 如果文献序号作为叙述文字的一部分, 需要临时将文献序号与 正文平排
%    \begin{macrocode}
\DeclareRobustCommand\inlinecite{\@inlinecite}
\def\@inlinecite#1{\begingroup\let\@cite\NAT@citenum\citep{#1}\endgroup}
%    \end{macrocode}
% \end{macro}
%
% \begin{environment}{thebibliography}
% 参考文献列表格式: 宋体 10.5 磅, 行距 20 磅, 续行缩进两字, 左对齐.
% 本科生依然是小四宋体.
%    \begin{macrocode}
\ifsustech@graduate
  \renewcommand\bibfont{\fontsize{10.5\bp@}{20\bp@}\selectfont}
\fi
\setlength{\bibsep}{0\p@ \@plus.2\p@}
\setlength{\bibhang}{2\ccwd}
\renewcommand\@biblabel[1]{[#1]\hfill}
%    \end{macrocode}
%
% 为了将参考文献加入目录和 pdf 书签, 重新定义 \pkg{natbib} 的 \cmd{bibsection}
%    \begin{macrocode}
\renewcommand\bibsection{%
  \@mainmatterfalse
  \chapter{\bibname}%
  \@mainmattertrue
}
%    \end{macrocode}
% \end{environment}
%
% BibTeX 生成参考文献表的样式在 \file{bst} 文件中提供.
%
%
% \subsection{附录}
%
% \begin{environment}{acknowledgments}
% 定义了一个满足要求的致谢环境:
%    \begin{macrocode}
\newenvironment{acknowledgments}{%
  \ifsustech@graduate
    \chapter{\sustech@acknowledgmentsname}%
  \else
    \sustech@chapter{\sustech@acknowledgmentsname}%
    \ifsustech@tocloaded
      本科生需要手动将致谢置于目录前。\par
    \fi
  \fi
}{}
%    \end{macrocode}
% \end{environment}
%
% \begin{environment}{publications}
% 发表成果环境:
%    \begin{macrocode}
\newenvironment{publications}{\chapter{\sustech@publicationsname}}{}
%    \end{macrocode}
% \end{environment}
%
%
% \subsection{其他宏包的设置}
%
% 这些宏包并非格式要求, 但是为了方便同学们使用, 在这里进行简单设置.
%    \begin{macrocode}
\newcommand\sustech@atendpackage{\csname ctex_at_end_package:nn\endcsname}
%    \end{macrocode}
%
%
% \subsubsection{\pkg{hyperref} 宏包}
%
%    \begin{macrocode}
\sustech@atendpackage{hyperref}{
  \hypersetup{
    bookmarksnumbered  = true,
    bookmarksopen      = true,
    bookmarksopenlevel = 1,
    linktoc            = all,
    unicode            = true,
    psdextra           = true,
  }
%    \end{macrocode}
%
% 如果为 \opt{pdf} 样式, 设置 hyperlink 颜色
%    \begin{macrocode}
  \ifsustech@pdf
    \hypersetup{
      colorlinks = true,
      allcolors  = blue,
    }
  \else
    \hypersetup{hidelinks}
  \fi
%    \end{macrocode}
%
% 填写 PDF 元信息.
%    \begin{macrocode}
  \AtBeginDocument{%
    \ifsustech@chinese
      \hypersetup{
        pdftitle  = \sustech@title,
        pdfauthor = \sustech@author,
      }%
    \else
      \hypersetup{
        pdftitle  = \sustech@entitle,
        pdfauthor = \sustech@enauthor,
      }%
    \fi
  }
%    \end{macrocode}
%
% 添加 PDF 书签
%
%    \begin{macrocode}
  \newcounter{sustech@bookmarknumber}
  \renewcommand\sustech@pdfbookmark[1]{%
    \phantomsection
    \stepcounter{sustech@bookmarknumber}%
    \pdfbookmark[0]{#1}{ustcchapter.\theustc@bookmarknumber}%
  }
%    \end{macrocode}
%
% 在 PDF 字符串中去掉换行, 以减少 \pkg{hyperref} 的警告信息.
%    \begin{macrocode}
  \pdfstringdefDisableCommands{
    \let\\\@empty
    \let\hspace\@gobble
  }
%    \end{macrocode}
%
% \pkg{hyperref} 与 \pkg{unicode-math} 存在一些兼容性问题, 见
% \href{https://github.com/ustctug/ustcthesis/issues/223}{
%   ustctug/ustcthesis\#223} 和
% \href{https://github.com/ho-tex/hyperref/pull/90}{ho-tex/hyperref\#90}。
%    \begin{macrocode}
  \@ifpackagelater{hyperref}{2019/04/27}{}{%
    \g@addto@macro\psdmapshortnames{\let\mu\textmugreek}%
  }
%    \end{macrocode}
%
% 设置中文的 \cs{autoref}.
% \footnote{\url{https://tex.stackexchange.com/a/66150/82731}}
%    \begin{macrocode}
  \ifsustech@chinese
    \def\equationautorefname~#1\null{公式~(#1)\null}
    \def\footnoteautorefname{脚注}
    \def\itemautorefname~#1\null{第~#1~项\null}
    \def\figureautorefname{图}
    \def\tableautorefname{表}
    \def\partautorefname~#1\null{第~#1~部分\null}
    \def\appendixautorefname{附录}
    \def\chapterautorefname~#1\null{第~#1~章\null}
    \def\sectionautorefname~#1\null{第~#1~节\null}
    \def\subsectionautorefname~#1\null{第~#1~小节\null}
    \def\subsubsectionautorefname~#1\null{第~#1~小小节\null}
    \def\paragraphautorefname~#1\null{第~#1~段\null}
    \def\subparagraphautorefname~#1\null{第~#1~小段\null}
    \def\theoremautorefname{定理}
    \def\HyRef@autopageref#1{\hyperref[{#1}]{第~\pageref*{#1} 页}}
  \fi
}
%    \end{macrocode}
%
%
% \subsubsection{\pkg{amsthm} 宏包}
%
%    \begin{macrocode}
\sustech@atendpackage{amsthm}{
  \newtheoremstyle{ustcplain}
    {}{}
    {}{2\ccwd}
    {\bfseries}{}
    {\ccwd}{}
  \theoremstyle{ustcplain}
%    \end{macrocode}
% 定义新的定理
%    \begin{macrocode}
  \ifsustech@chinese
    \newcommand\sustech@assertionname{断言}
    \newcommand\sustech@assumptionname{假设}
    \newcommand\sustech@axiomname{公理}
    \newcommand\sustech@corollaryname{推论}
    \newcommand\sustech@definitionname{定义}
    \newcommand\sustech@examplename{例}
    \newcommand\sustech@lemmaname{引理}
    \newcommand\sustech@proofname{证明}
    \newcommand\sustech@propositionname{命题}
    \newcommand\sustech@remarkname{注}
    \newcommand\sustech@theoremname{定理}
  \else
    \newcommand\sustech@assertionname{Assertion}
    \newcommand\sustech@assumptionname{Assumption}
    \newcommand\sustech@axiomname{Axiom}
    \newcommand\sustech@corollaryname{Corollary}
    \newcommand\sustech@definitionname{Definition}
    \newcommand\sustech@examplename{Example}
    \newcommand\sustech@lemmaname{Lemma}
    \newcommand\sustech@proofname{Proof}
    \newcommand\sustech@propositionname{Proposition}
    \newcommand\sustech@remarkname{Remark}
    \newcommand\sustech@theoremname{Theorem}
  \fi
  \newtheorem{theorem}             {\sustech@theoremname}    [chapter]
  \newtheorem{assertion}  [theorem]{\sustech@assertionname}
  \newtheorem{axiom}      [theorem]{\sustech@axiomname}
  \newtheorem{corollary}  [theorem]{\sustech@corollaryname}
  \newtheorem{lemma}      [theorem]{\sustech@lemmaname}
  \newtheorem{proposition}[theorem]{\sustech@propositionname}
  \newtheorem{assumption}          {\sustech@assumptionname} [chapter]
  \newtheorem{definition}          {\sustech@definitionname} [chapter]
  \newtheorem{example}             {\sustech@examplename}    [chapter]
  \newtheorem*{remark}             {\sustech@remarkname}
%    \end{macrocode}
% \pkg{amsthm} 单独定义了 proof 环境, 这里重新定义以满足格式要求.
% 原本模仿 \pkg{amsthm} 写成 |\item[\hskip\labelsep\hskip2\ccwd #1\hskip\ccwd]|,
% 但是却会多出一些间隙.
%    \begin{macrocode}
  \renewenvironment{proof}[1][\proofname]{\par
    \pushQED{\qed}%
    \normalfont \topsep6\p@\@plus6\p@\relax
    \trivlist
      \item\relax\hskip2\ccwd
      \textbf{#1}
      \hskip\ccwd\ignorespaces
    }{%
    \popQED\endtrivlist\@endpefalse
  }
  \renewcommand\proofname\sustech@proofname
}
%    \end{macrocode}
%
%
% \subsubsection{\pkg{algorithm2e} 宏包}
%
% 按章节编号.
%    \begin{macrocode}
\PassOptionsToPackage{algochapter}{algorithm2e}
%    \end{macrocode}
%
%    \begin{macrocode}
\sustech@atendpackage{algorithm2e}{
  \ifsustech@chinese
    \SetAlgorithmName{算法}{算法}{算法清单}
  \else
    \SetAlgorithmName{Algorithm}{Algorithm}{List of Algorithms}
  \fi
%    \end{macrocode}
%
% 设置算法环境的格式.
%    \begin{macrocode}
  \SetAlCapSkip{6\bp@}
  \SetAlCapFnt{\small}
  \SetAlCapNameFnt{\small}
  \ifsustech@graduate
    \SetAlCapNameSty{textbf}
  \fi
  \SetAlgoCaptionSeparator{\unskip\hspace*{\ccwd}}
%    \end{macrocode}
%
% 设置算法清单的格式
%    \begin{macrocode}
  \renewcommand\listofalgocfs{%
    \ifsustech@graduate
      \cleardoublepage
    \fi
    \sustech@chapter{\listalgorithmcfname}%
    \@starttoc{loa}%
  }
  \titlecontents{algocf}
    [2.3em]{\normalsize}
    {\contentslabel{2.3em}}{}
    {\sustech@leaders\contentspage}
  \contentsuse{algocf}{loa}
}
%    \end{macrocode}
%
%
% \subsubsection{\pkg{nomencl} 宏包}
%
%    \begin{macrocode}
\sustech@atendpackage{nomencl}{
  \let\nomname\sustech@notationname
  \def\thenomenclature{%
    \ifsustech@graduate
      \cleardoublepage
      \pagestyle{headings}%
    \fi
    \sustech@chapter{\sustech@notationname}%
    \nompreamble
    \list{}{%
      \labelwidth\nom@tempdim
      \leftmargin\labelwidth
      \advance\leftmargin\labelsep
      \itemsep\nomitemsep
      \let\makelabel\nomlabel}}
  \def\endthenomenclature{%
    \endlist
    \nompostamble
    \ifsustech@graduate
      \clearpage
      \pagestyle{plain}%
    \fi
  }
}
%    \end{macrocode}
%
%
% \subsubsection{\pkg{siunitx} 宏包}
%
%    \begin{macrocode}
\sustech@atendpackage{siunitx}{
  \sisetup{
    group-minimum-digits = 4,
    separate-uncertainty = true,
    inter-unit-product   = \ensuremath{{}\cdot{}},
  }
  \ifsustech@chinese
    \sisetup{
      list-final-separator = { 和 },
      list-pair-separator  = { 和 },
      range-phrase         = {~},
    }
  \fi
}
%    \end{macrocode}
%
% \Finale
\endinput
